\section{Durchführung}
\label{sec:Durchführung}

%Was wurde gemessen bzw. welche Größen wurden variiert?

\subsection{Amplitudenmodulation}
Zuerst wird eine Amplitudenmodulation mit Hilfe eines Ringmodulators gemäß Abbildung~\ref{fig:ring} erzeugt, wobei das resultierende Signal mit einem 
Oszilloskop dargestellt wird. Das Bild der entstehenden Schwebung wird als PNG abgespeichert. Außerdem wird das Spektrum der Schwebung über den 
Spektrumanalysator aufgenommen und als CSV-Datei gespeichert. Danach wird eine AM mit Hilfe einer Diode gemäß Abbildung~\ref{fig:diode_am} erzeugt. Auch hier wird 
das resultierende Signal am Oszilloskop abgespeichert und das Spektrum aufgenommen. \newline\newline 

\subsection{Frequenzmodulation}
Zur Erzeugung einer Frequenzmodulation wird die in Abbildung~\ref{fig:fm} dargestellte Schaltung verwendet. Das resultierende Signal wird am 
Oszilloskop beobachtet. Zunächst wird mit manuellem Triggern ein stabiles Bild der modulierten Trägerschwingung eingestellt und als PNG-Datei gespeichert. 
Anschließend wird auf den Eingang des Signals getriggert, wodurch eine phasenverschobene Überlagerung auftritt, die als periodische Verschmierung der 
Signalkurve sichtbar ist. Der Bildausschnitt wird so gewählt, dass die Verschmierung maximal ausgeprägt ist. 

\subsection{Demodulation}
Zur Demodulation wird zunächst ein phasenempfindlicher Gleichrichter auf Basis eines Ringmodulators gemäß 
Abbildung~\ref{fig:demod_ring} aufgebaut. Die beiden Eingänge des Ringmodulators werden mit Signalen gleicher Frequenz und variabler Phasenverschiebung verbunden.
Das Ausgangssignal wird über einen Tiefpass gefiltert, sodass eine Gleichspannung entsteht, die proportional zum Kosinus der Phasendifferenz zwischen den beiden 
Eingängen ist. Zur Überprüfung wird ein Multimeter an den Ausgang des Tiefpasses angeschlossen und die Ausgangsspannung für 15 verschiedene 
Phasenverschiebungen im Bereich von $0^\circ$ bis $180^\circ$ gemessen. Die Phasenverschiebung wird dabei durch entsprechende Einstellung der Wellenlänge am 
HF-Generator variiert.
\begin{figure}
    \centering
    \includegraphics[width=0.7\textwidth]{data/demod_ring.png}
    \caption{Schaltung zur synchronen Demodulation eines AM Signals mit einem Ringmodulator.}
    \label{fig:demod_ring}
\end{figure}

Für eine weitere Methode der Demodulation wird ein AM Signal mit Trägerabstrahlung erzeugt, indem dem Ausgang eines Ringmodulators das Trägersignal über einen 
Leistungsteiler wieder hinzuaddiert wird, wie in Abbildung~\ref{fig:demod_2} dargestellt. Dieses Signal wird anschließend mit einem weiteren Ringmodulator und einem nachgeschalteten Tiefpass demoduliert. 
Die Modulationsspannung wird zusätzlich auf den zweiten Kanal des Oszilloskops gelegt. Der zeitliche Verlauf des Signals hinter dem 
Ringmodulator sowie am Ausgang des Tiefpasses wird jeweils mit dem Oszilloskop aufgezeichnet und als PNG-Datei gespeichert.
\begin{figure}
    \centering
    \includegraphics[width=0.7\textwidth]{data/demod_2.png}
    \caption{Schaltung zur Demodulation eines AM Signals mit Trägerabstrahlung.}
    \label{fig:demod_2}
\end{figure}
