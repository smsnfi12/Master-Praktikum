\section{Theorie}
\label{sec:Theorie}

\subsection{Das Konzept einer Modulation}
Das Übertragen niederfrequenter Strahlung über lange Reichweiten ist in der Praxis nicht praktisch, da das Empfangen nur über Antennen möglich ist, 
deren benötigten Maße wiederum mit der Wellenlänge der Strahlung skaliert. Außerdem kommt es beim Übertragen mehrerer Signale im gleichen Frequenzbereich 
zu Überlagerungen und damit zu Störungen der einzelnen Signale. Das Konzept der Modulation besteht nun darin, ein hochfrequentes Trägersignal, welches stabil
und effizient übertragen werden kann, mit einem niederfrequenten Signal, welches die zu übertragene Information trägt, zu verknüpfen. Dadurch entsteht ein 
kombiniertes Signal, welches niederfrequente Wellen implizit auf effiziente und stabile Weise übertragen kann. 

\subsection{Amplitudenmodulation}
Eine der möglichen Methoden, ein Trägersignal zu modulieren, ist die Amplitudenmodulation\,(AM). Durch eine geeignete Schaltung, wie in Unterabschnitt[] beschrieben 
wird, wird eine modulierte Spannung gemäß 
\begin{equation}
    u_{\mathrm{AM}}(t)
    =
    U_T \bigl( 1 + m \cos(\omega_M t) \bigr)\cos(\omega_T t)
\end{equation}
erzeugt. Hierbei bezeichnet $U_T$ die Trägeramplitude, $\omega_T$ die Trägerfrequenz,
$\omega_M$ die Modulationsfrequenz und $m$ den Modulationsgrad. Wie einleitend erwähnt, muss dabei die Voraussetzung $\omega_M \ll \omega_T$ erfüllt sein. 
Druch das Produkt der beiden zeitabhängigen Terme ergibt sich ein charakteristisches Spektrum. Der Term $ U_T \cos(\omega_T t)$ erzeugt die Trägerlinie bei 
$\omega_T$, welche keine Informationen enthält und damit ausschließlich als Referenz- und Transportfrequenz dient. Der modulierte Anteil 
$m\cos(\omega_M t)\cos(\omega_T t)$ erzeugt die beiden Seitenlinien bei $(\omega_T \pm \omega_M)$. Dieser Teil des Spektrums enthält die vollständige Information
des Modulationssignals, codiert durch die Verschiebung $\pm \omega_M$ im Frequenzraum. Die Amplitude der Seitenlinien ist proportional zu $U_T$ und $m$. Daher kann 
durch Messung der jeweiligen Amplituden der Modulationsgrad über 
\begin{equation}
    m = \frac{U_\mathrm{max} - U_\mathrm{min}}{U_\mathrm{max} + U_\mathrm{min}}
\end{equation}
bestimmt werden. \newline\newline
Die experimentelle Erzeugung einer AM erfordert eine Schaltung, die ein Produkt aus Träger- und Modulationssignal kreiert. Dies impliziert eine Nichtlinearität, was 
durch eine Diode realisiert wird. Die Kennlinie einer Diode ist im Allgemeinen exponentiell von $U$ abhängig, kann aber für kleine Wechselspannungen um einen 
Arbeitspunkt über 
\begin{equation}
    I(U) \approx a_1U + a_2U^2 + \mathcal{O}(U^3)
\end{equation}
entwickelt werden. Wird an eine Diode eine Spannung der Form 
\begin{equation}
    U(t) = U_T\cos(\omega_T\,t) + U_M\cos(\omega_M\,t)
\end{equation}
angelegt, entstehen durch den $U^2$-Term die entsprechenden Produkte der beiden Signale, welche zu den Seitenlinien bei $(\omega_T \pm \omega_M)$ führen. 
Eine entsprechende Skizze ist in Abbildung~\ref{fig:diode_am} zu sehen. In dieser Schaltung werden zwar die informationshaltigen Seitenbänder erzeugt, aber auch 
die Trägerlinie und weitere unerwünschte Mischprodukte durch die $\mathcal{O}(U^3)$-Terme. Eine effizientere Schaltung erfolgt über einen Ringmodulator, wie er in 
Abbildung~\ref{fig:ring} dargestellt ist. Das Trägersignal wird so an den Diodenring gekoppelt, dass sich aufgrund seiner Symmetrie die Beiträge am Ausgang 
paarweise aufheben. Ohne ein Modulationssignal wird also kein Signal übertragen, was die Grundlage für die Trägerunterdrückung darstellt. Die Modulationsspannung
wird differentiell an zwei gegenüberliegenden Knoten angelegt. Durch das wechselnde Vorzeichen des Trägers werden abwechselnd unterschiedliche Diodenpaare leitend, 
sodass die Modulationsspannung periodisch mit positivem oder negativem Vorzeichen an den Ausgang übertragen wird. Dieses Signal am Ausgang ist dann ein Produkt aus
Träger- und Modulationssignal, während das reine Trägersignal unterdrückt wird. 

\begin{figure}[H]
    \centering
    \includegraphics[width=0.4\textwidth]{data/diode_am.png}
    \caption{Skizze einer einfachen AM durch einzelne Diode.}
    \label{fig:diode_am}
\end{figure}

\begin{figure}[H]
    \centering
    \includegraphics[width=0.7\textwidth]{data/ringmod.png}
    \caption{Skizze einer AM über einen Ringmodulator.}
    \label{fig:ring}
\end{figure}

\subsection{Frequenzmodulation}
Eine weitere Möglichkeit der Modulation ist die Frequenzmodulation\,(FM), bei der nicht die Amplitude, sondern die momentane Frequenz moduliert wird. 
Hierbei ist die modulierte Amplitude optimalerweise konstant, die Information wird also nicht über die Hüllkurve des Gesamtsignals, sondern über die zeitliche 
Änderung dessen Phase codiert. Ein FM-Signal lässt sich im Allgemeinen in der Form 
\begin{equation}
    U_\mathrm{FM}(t) = U_0 \cos\bigl( \omega_T\,t + \phi(t) \bigr)
\end{equation}
beschreiben. Dabei ist die momentane Kreisfrequenz durch $\omega(t) = \omega_T + \dot{\phi}(t)$ definiert, wobei $\phi(t)$ die zeitabhängige Phasenfunktion ist. 
Für ein sinusförmiges Modulationssignal mit $\phi(t) = \beta\sin(\omega_M\,t)$ ergibt sich dann eine momentane Kreisfrequenz von 
\begin{equation}
    \omega(t) = \omega_T + \beta \omega_m \cos(\omega_M\,t)\,,
\end{equation}
mit dem FM-Modulationsindex $\beta$, der das Verhältnis von maximalem Frequenzhub zur Modulationsfrequenz angibt.\newline
Im Spektrum tritt hier ebenfalls eine Trägerlinie bei $\omega_T$ auf, allerdings theoretisch undendlich viele Seitenlinien bei $\omega_T \pm n \omega_M$ mit 
$n \in \mathbf{N}$, deren Amplituden von $\beta$ abhängig sind, aber mit wachsendem $n$ typischerweise abnehmen. Eine vollständige Beschreibung der Information 
erfolgt dann über Besselfunktionen. In diesem Versuch wird allerdings mit $\beta \ll 1$ gearbeitet, wodurch die Besselfunktion $J_{n\geq2}(\beta) \approx 0$, 
und folglich das Spektrum der Seitenlinien effektiv auf zwei Linien $(\omega_T \pm \omega_M)$ reduziert werden kann. Die entsprechende Modulation in dieser 
Näherung wird als schmalbandige FM\,(NBFM) bezeichnet. 

\subsection{Demodulation}
