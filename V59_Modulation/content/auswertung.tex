\section{Auswertung}
\label{sec:Auswertung}
Alle Berechnungen werden mit dem Programm \glqq Numpy" \cite{numpy}, die Unsicherheiten mit dem Modul \glqq Uncertainties" \cite{uncertainties}, die Ausgleichsrechnungen mit dem Modul \glqq Scipy" \cite{scipy} durchgeführt und die grafischen Darstellungen über das Modul \glqq Matplotlib" \cite{matplotlib} erstellt.

\subsection{Amplitudenmodulation}
Im Folgenden wird die Amplitudenmodulation untersucht. Zuerst wird die Amplitudenmodulation mit Trägerunterdrückung betrachtet, danach die Amplitudenmodulation mit Trägerabstrahlung.
\subsubsection{Trägerunterdrückung}
Die Amplitudenmodulation mit Trägerunterdrückung wird mit dem in Abschnitt [???] beschriebenen Versuchsaufbau durchgeführt.
Dazu werden folgende Werte für die Träger- und Modulationsspannung  eingestellt:
\begin{align*}
    U_\text{T} &= 1 ~ \text{V}, \\
    U_\text{M} &= 0,1 ~ \text{V}.
\end{align*}
Die jeweiligen Frequenzen betragen:
\begin{align*}
    f_\text{T} &= 2 ~ \text{MHz}, \\
    f_\text{M} &= 100 ~ \text{kHz}.
\end{align*}
In der Abbildung \ref{fig:am_both} ist das zeitliche Spektrum dargestellt.\\
\begin{figure}
    \centering
    \includegraphics[width=0.7\textwidth]{Plots/AM_Trägerunterdrückung.pdf}
    \caption{Zeitliches Spektrum der Amplitudenmodulation mit Trägerunterdrückung.}
    \label{fig:am_both}
\end{figure}
In der Abbildung ist eine Schwebung zu erkennen, die durch die Überlagerung der beiden Seitenbänder entsteht.
Die Hüllkurve liegt nicht symmetrisch um die Nulllinie ($U=0$ V), sondern ist leicht nach unten verschoben.
Aufgrund der Unterdrückung des Trägers verschwindet das Signal in den Hüllkurven-Minima nahezu vollständig.\\
\begin{figure}[H]
    \centering
    \includegraphics[width=0.7\textwidth]{Plots/02.pdf}
    \caption{Frequenzspektrum der Amplitudenmodulation mit Trägerunterdrückung.}
    \label{fig:am_freq1}
\end{figure}
Die Aufnahme des Frequenzspektrums über den Spektrumanalysator ist in Abbildung \ref{fig:am_freq1} dargestellt.
Die beiden Seitenbänder sind deutlich zu erkennen, wobei der Träger bei der Frequenz $f_\text{T} = 2$ MHz nahezu vollständig unterdrückt ist.
Die gemessenen Frequenzen der Seitenbänder stimmen mit den eingestellten Frequenzen überein und sind in der Tabelle \ref{tab:am_freq1} aufgelistet.
Die Seitenbänder sind symmetrisch um die Trägerfrequenz angeordnet und haben annähernd die gleiche Amplitude.
\begin{table}
    \centering
    \caption{Seitenbänder bei der Amplitudenmodulation mit Trägerunterdrückung.}
    \label{tab:am_freq1}
    \begin{tabular}{c | c c |c}
        \toprule
        & erwartete Frequenz & gemessene Frequenz & Amplitude \\
        & $f$ / MHz & $f$ / MHz & $A$ / dBm \\
        \midrule
        Peak 1 & $f_\text{T}-f_\text{M}$ = 1,9 & 1,9 & -28,10\\
        Peak 2 & $f_\text{T}+f_\text{M}$ = 2,1 & 2,1 & -28,11\\
        \bottomrule
    \end{tabular}
\end{table}

\subsubsection{Trägerabstrahlung}
Die Amplitudenmodulation mit Trägerabstrahlung wird mit in der Tabelle \ref{tab:am_values} aufgelisteten Werten durchgeführt.
\begin{table}
    \centering
    \caption{Eingestellte Werte für die Amplitudenmodulation mit Trägerabstrahlung.}
    \label{tab:am_values}
    \begin{tabular}{c | c c}
        \toprule
        & $U$ / V & $f$ / MHz \\
        \midrule
        Trägersignal & 1,0 & 5,0\\
        Modulationssignal & 0,8 & 0,2\\
        \bottomrule
    \end{tabular}
\end{table}
In der Abbildung \ref{fig:am_carrier} ist das zeitliche Spektrum dargestellt.\\
\begin{figure}[H]
    \centering
    \includegraphics[width=0.7\textwidth]{Plots/AM_Trägerabstrahlung.pdf}
    \caption{Zeitliches Spektrum der Amplitudenmodulation mit Trägerabstrahlung.}
    \label{fig:am_carrier}
\end{figure}
In der Abbildung ist eine deutliche Amplitudenmodulation zu erkennen, bei der die Hüllkurve das Trägersignal umschließt.
Im Gegensatz zur Amplitudenmodulation mit Trägerunterdrückung verschwindet das Signal in den Hüllkurven-Minima nicht, da das Trägersignal vorhanden ist.\\
Zur Bestimmung des Modulationsgrades $m$ werden die Maximal- und Minimalwerte der Hüllkurve für 3 Maxima und 4 Minima entnommen.
Dabei wird für das jeweilige Maximum und Minimum der Unterschied zwischen dem höchsten und niedrigsten Werten des Extremums gebildet.
Zur Minimierung von Ablesefehlern werden die gemessenen Werte gemittelt.
Die Ergebnisse für das Maximum $U_\text{max}$ und Minimum $U_\text{min}$ sind in der Tabelle \ref{tab:am_mod} aufgelistet.\\
\begin{table}
    \centering
    \caption{Abgelesene Maximal- und Minimalwerte der Hüllkurve zur Bestimmung des Modulationsgrades $m$.}
    \label{tab:am_mod}      
    \begin{tabular}{c c | c c}
        \toprule
        \multicolumn{2}{c}{Maximum} & \multicolumn{2}{c}{Minimum} \\
        \midrule
        $t$ / $\mu$s & $U_\text{max}$ / mV & $t$ / $\mu$s & $U_\text{min}$ / mV \\
        \hline
        -5 & 10,86 & -7,5 & 5,63 \\
        0 & 10,86 & -2,5 & 5,26 \\
        5 & 10,86 & 2,5 & 5,55 \\
        - & - & 7,5 & 5,66\\
        \midrule
        & 18,86 $\pm$ 0,02 & & 5,52 $\pm$ 0,09 \\
        \bottomrule
    \end{tabular}
\end{table}
Daraus ergibt sich für den Modulationsgrad $m$ nach Gleichung [???]:
\begin{equation*}
    m = 0,326 \pm 0,008.
\end{equation*}



In der Abbildung \ref{fig:am_freq2} ist das Frequenzspektrum dargestellt.\\
\begin{figure}[H]
    \centering
    \includegraphics[width=0.7\textwidth]{Plots/04.pdf}
    \caption{Frequenzspektrum der Amplitudenmodulation mit Trägerabstrahlung.}
    \label{fig:am_freq2}
\end{figure}
Der dominierende Peak bei 5 MHz entspricht dem Trägersignal.
Die beiden Seitenbänder sind symmetrisch um den Träger angeordnet, die genauen Werte sind in der Tabelle \ref{tab:am_freq2} aufgelistet.\\
\begin{table}
    \centering
    \caption{Amplitudenmodulation mit Trägerabstrahlung. Die gemessenen und erwarteten Frequenzen des Trägersignals, der Seitenbänder, sowie der ersten beiden Oberwellen sind dargestellt.}
    \label{tab:am_freq2}
    \begin{tabular}{c | c c |c}
        \toprule
        & erwartete Frequenz & gemessene Frequenz & Leistungspegel \\
        & $f$ / MHz & $f$ / MHz & $L$ / dBm \\
        \midrule
        Trägersignal & 5,0 & 5,0 & -60 $\pm$ 1\\
        \hline
        \midrule
        Seitenbänder & $f_\text{T} \pm f_\text{M}$ & & \\
        \hline
        Peak 1 & 5,2 & 5,2 & -70 $\pm$ 1\\
        Peak 2 & 4,8 & 4,8 & -70 $\pm$ 1\\
        \hline
        \midrule
        Oberwellen & $f_\text{T} \pm f_\text{M}$ & & \\
        \hline
        Peak 3 & 5,4 & 5,4 & -74 $\pm$ 1\\
        Peak 4 & 4,6 & 4,6 & -74 $\pm$ 1\\
        \hline
        \bottomrule
    \end{tabular}
\end{table}
Aufgrund der Nichtlinearität der Diode treten zusätzlich zu den Seitenbändern auch Oberwellen auf.
Die ersten beiden Oberwellen sind deutlich in der Abbildung \ref{fig:am_freq2} sichtbar.
Sowohl die Seitenbänder als auch die Oberwellen sind symmetrisch um den Träger positioniert.
Die gemessenen Frequenzen stimmen mit den Erwarteten überein.\\
Der Modulationsgrad $m$ kann auch wie in Gleichung [???] über das Verhältnis der Amplituden der Seitenbänder zum Trägersignal bestimmt werden.
Die Amplituden werden aus den Leistungspegeln in der Tabelle \ref{tab:am_freq2} berechnet, wobei die Messunsicherheit aufgrund der Ablesegenauigkeit des Spektrumanalysators mit 1 dBm angenommen wird.
Der Leistungspegel wird in die Leistung umgerechnet über
\begin{equation*}
    P = 10^{\frac{L}{10}} \text{ mW}.
\end{equation*}
Die Spannungsamplitude ergibt sich dann zu
\begin{equation*}
    U = \sqrt{P \cdot R} = \sqrt{10^{\frac{L}{10}} \cdot R} ,
\end{equation*}
wobei der Widerstand $R = 50 ~ \Omega$ beträgt.
Es lässt sich daraus die Spannungsamplitude des Trägersignals $U_\text{T}$ und der Seitenbänder $U_\text{SB}$ bestimmen:
\begin{align*}
    U_\text{T} &= 0,22 \pm 0,03 ~ \text{mV}, \\
    U_\text{SB} &= (0,071 \pm 0,008) ~ \text{mV}.
\end{align*}
Damit ergibt sich für den Modulationsgrad $m$:
\begin{equation*}
    m = 0,32 \pm 0,05.
\end{equation*}
Die beiden bestimmten Werte für den Modulationsgrad $m$ stimmen innerhalb der Messunsicherheiten überein.



\subsection{Frequenzmodulation}
Das frequenzmoduliertes Signal besteht aus einem Träger- und Modulationssignal mit folgenden charakteristischen Werten:
\begin{align*}
    U_\text{T} = 1 ~ \text{V},~&   ~f_\text{T} = 5 ~ \text{MHz}, \\
    U_\text{M} = 0,2 ~ \text{V},~&  ~f_\text{M} = 200 ~ \text{kHz}.
\end{align*}
Zur Gewährleistung der $90^\circ$ Phasenverschiebung zwischen Träger- und Modulationssignal wird die Laufzeit des Trägersignals um
\begin{equation*}
    \Delta t = \frac{1}{4 f_\text{T}} = 50 ~ \text{ns}
\end{equation*}
verlängert, wobei $f_\text{T} = 5$ MHz die Frequenz des Trägersignals ist. 
In der Abbildung \ref{fig:fm_time} ist das zeitliche Spektrum der Frequenzmodulation dargestellt.\\
\begin{figure}[H]
    \centering
    \includegraphics[width=0.7\textwidth]{data/scope_4.png}
    \caption{Zeitliches Spektrum der Frequenzmodulation.}
    \label{fig:fm_time}
\end{figure}
In der Abbildung sind das Träger- und Modulationssignal zu erkennen.
Dabei ist die periodische Verschmierung des Trägersignals durch die Frequenzmodulation sichtbar.\\
Der Frequenzhub $\Delta f$ wird aus den Randwerten der Verschmierung über $f_\text{min} = 1/t_\text{min}$ und $f_\text{max} = 1/t_\text{max}$ bestimmt.
Die gemessenen Zeitpunkte $t_1$ und $t_2$, sowie die entsprechenden Werte für $\Delta f=|t_1-t_2|/2$ sind in der Tabelle \ref{tab:verschmierung} aufgelistet.\\
\begin{table}[h]
\centering
\caption{Randwerte der Verschmierung zur Bestimmung des Frequenzhubs $\Delta f$.}
\begin{tabular}{r r c}
\hline
$t_1$ /ns & $t_2$ /ns & $\Delta f$ /MHz\\ 
\hline
-220   & -244    & 0,22\\
-326   & -350    & 0,10\\
-420,5 & -441,5  & 0,06\\
\bottomrule
\end{tabular}
\label{tab:verschmierung}
\end{table}
Aus dem Mittelwert der berechneten $\Delta f$ ergibt sich für den Frequenzhub $\Delta f$
\begin{equation*}
    \Delta f =  0,13 \pm 0,05 ~ \text{MHz}.
\end{equation*} 
Der Modulationsgrad $m$ wird über Gleichung [???] bestimmt:
\begin{equation*}
    m = 0,64 \pm 0,25.
\end{equation*}

\subsection{Demodulation}
Für alle folgenden Messungen zur Demodulation werden das Träger- und Modulationssignal wie folgt eingestellt:
\begin{align*}
    U_\text{T} = 1 ~ \text{V},~&   ~f_\text{T} = 5 ~ \text{MHz}, \\
    U_\text{M} = 0,2 ~ \text{V},~&  ~f_\text{M} = 200 ~ \text{kHz}.
\end{align*}

\subsubsection{Phasenempfindlicher Gleichrichter}
Zur Untersuchung deer Phasenverschiebung wird das Demodulatorsignal bei verschiedenen Phaseneinstellungen aufgenommen.
In der Tabelle \ref{tab:phase} sind die Messwerte bei den jeweiligen Phaseneinstellungen aufgelistet.\\
\begin{table}[h]
    \centering
    \caption{Messwerte des Demodulatorsignals bei verschiedenen Phaseneinstellungen.}
    \label{tab:phase}
    \begin{tabular}{c | c c}
        \toprule
        Zeitverschiebung & Phasenverschiebung& Spannung \\
        $\Delta t$ / ns & $\phi$ / & $U$ / mV \\
        \midrule
        10  & 0,31 & 1,625   \\
16  & 0,50 & 10,375  \\
20  & 0,63 & 19,875  \\
26  & 0,82 & 33,125  \\
32  & 1,01 & 45,875  \\
38  & 1,20 & 57,375  \\
44  & 1,38 & 69,875  \\
50  & 1,57 & 81,875  \\
56  & 1,76 & 92,375  \\
62  & 1,95 & 102,375 \\
70  & 2,20 & 112,500 \\
78  & 2,45 & 117,125 \\
86  & 2,70 & 118,625 \\
96  & 3,01 & 117,125 \\
100 & 3,14 & 115,125 \\
        \bottomrule
    \end{tabular}
\end{table}
In der Abbildung \ref{fig:phase} sind die Messwerte grafisch dargestellt.\\
\begin{figure}[H]
    \centering
    \includegraphics[width=0.7\textwidth]{build/Demodulation.pdf}
    \caption{Demodulatorspannung in Abhängigkeit der Phasenverschiebung.}
    \label{fig:phase}
\end{figure}
Ein kosinusförmiger Verlauf der Demodulatorspannung in Abhängigkeit der Phasenverschiebung ist leicht zu erkennen.
Zu beachten ist der rechte Grenzwert bei $\phi = \pi$, da hier die Phasenverschiebung wieder sinkt.

\subsubsection{Ringmodulator und Tiefpass}
Im folgenden wird die Demodulation mit dem Ringmodulator und Tiefpass untersucht.
In der Abbildung \ref{fig:ringmod} ist das zeitliche Spektrum des demodulierten Signals dargestellt.\\
\begin{figure}[H]
    \centering
    \includegraphics[width=0.7\textwidth]{data/scope_8.png}
    \caption{Zeitliches Spektrum des demodulierten Signals mit Ringmodulator und Tiefpass. Gelb: demoduliertes Trägersignal, Grün: Modulationssignal.}
    \label{fig:ringmod}
\end{figure}
Wie es aus der Abbildung zu erkennen ist, unterscheiden sich die beiden Signale nur in Amplitude und Phase.
Die Trägerschwingung hat eine Hüllkurve, die ungefähr mit der des Modulationssignals mitläuft.
Dies bestätigt die erfolgreiche Demodulation des Amplitudenmodulierten Signals.


\subsubsection{Gleichrichter und Tiefpass}
In der Abbildung \ref{fig:gleichrichter} ist das zeitliche Spektrum des demodulierten Signals mit Gleichrichter und Tiefpass dargestellt.\\
\begin{figure}[H]
    \centering
    \includegraphics[width=0.7\textwidth]{data/scope_10.png}
    \caption{Zeitliches Spektrum des demodulierten Signals mit Gleichrichter und Tiefpass. Grün: demoduliertes Trägersignal, Gelb: Modulationssignal.}
    \label{fig:gleichrichter}
\end{figure}
Auch hier ist die erfolgreiche Demodulation des Amplitudenmodulierten Signals zu erkennen.
Das demodulierte Signal hat deutliche nicht sinusförmige Spitzen, was auf die Nichtlinearität des Aufbaus zurückzuführen ist.
Beide Signale haben dieselbe Periodendauer und die Maxima treten ungefähr zur gleichen Zeit auf.
