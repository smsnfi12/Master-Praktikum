\section{Diskussion}
\label{sec:Diskussion}
\subsection{Amplitudenmodulation}
Die Durchführung von Amplitudenmodulationen mit und ohne Trägerunterdrückung ist gelungen.
In den Abbildungen \ref{fig:am_both} und \ref{fig:am_carrier} stellen die gemessenen zeitlichen Spektra der Amplitudenmodulationen graphisch dar.
In beiden Fäller sind die Hüllenkurven gut zu erkennen und stimmen mit den theoretisch erwarteten Verläufen überein.
Im Falle der Amplitudenmodulation mit Trägerunterdrückung sehen die Maxima der Hüllenkurve etwas spitzer aus als bei der Amplitudenmodulation ohne Trägerunterdrückung.
Dies könnte auf eine nicht optimale Einstellung des Modulationssignales, sowie auf Nichtlinearitäten bzw. Toleranzen der verwendeten Bauteile zurückzuführen sein.\\
Die Frequenzspektren in den Abbildungen \ref{fig:am_freq1} und \ref{fig:am_freq2} zeigen die erwarteten Seitenbänder und Trägersignale.
Die Seitenbänder sind in beiden Fällen symmetrisch um die Trägerfrequenz angeordnet.
Im Frequenzspektrum der Amplitudenmodulation mit Trägerabstrahlung ist das Trägersignal deutlich stärker ausgeprägt als die Seitenbänder.
Dies ist darauf zurückzuführen, dass das Trägersignal bei dieser Modulationsart nicht unterdrückt wird.
Die ersten beiden Oberwellen sind ebenfalls gut zu erkennen.
Aufgrund des Rauschens und der begrenzten Auflösung sind die höheren Oberwellen nur schwer zu identifizieren.
Die gemessenen Verhältnisse der Seitenbänder zum Trägersignal stimmen gut mit den theoretisch erwarteten Werten überein.\\
Der Modulationsgrad des Signals mit Trägerabstrahlung wird anhand des zeitlichen Spektrums und des Frequenzspektrums bestimmt.
Das zeitliche Spektrum liefert
\begin{equation*}
    m_\text{zeit} = 0,326 \pm 0,008,
\end{equation*}
wobei der Modulationsgrad bestimmt anhand des Frequenzspektrums
\begin{equation*}
    m_\text{freq} = 0,32 \pm 0,05
\end{equation*}
beträgt.
Die beiden Werte stimmen gut überein und liegen im Bereich der Messunsicherheiten.
Der erste Modulationssignal $m_\text{zeit}$ ist dabei etwas genauer bestimmt, was auf die höhere Auflösung des Oszilloskops zurückzuführen ist.
Die bestimmte Werte liegen im erwarteten Bereich für die eingestellte Spannungen des Träger- und Modulationssignals.\\
Insgesamt sind die Ergebnisse der Amplitudenmodulation zufriedenstellend und stimmen gut mit den theoretischen Erwartungen überein.

\subsection{Frequenzmodulation}
Bei dem frequenzmodulierten Signal ist die Verschmierung der Trägerfrequenz gut zu erkennen, wie in Abbildung \ref{fig:fm_time} dargestellt.
Anhand der Verschmierung wird der Frequenzhub zu
\begin{equation*}
    \Delta f = 0,13 \pm 0,05 ~ \text{MHz}
\end{equation*}
bestimmt.
Der damit bestimmte Modulationsgrad beträgt
\begin{equation*}
    m = 0,6 \pm 0,2.
\end{equation*}
Der Modulationsgrad $m$ ist größer als im Fall der Amplitudenmodulation.
Die große Unsicherheit ist auf die schwierige Ablesbarkeit der Verschmierung im zeitlichen Spektrum zurückzuführen.\\
Im folgendem Versuch wird die sogennante Schmalband-Frequenzmodulation betrachtet.
Die Näherung ist allerdings mit den eingestellten Parametern und dem bestimmten Modulationsgrad nicht gerechtfertigt, da $m$ nicht viel kleiner als 1 ist.

\subsection{Demodulation}
In der Demodulation mit dem phasenempfindlichen Gleichrichter wird die Phasenabhängigkeit des Signals untersucht.
Die Ergebnisse in Abbildung \ref{fig:phase} zeigen einen Teil der theoretisch erwarteten Sinuskurve.
Es ist schwierig auf die exakte Form der Kurve zu schließen, da Messpunkte nur im Bereich von $0$ bis $\pi$ aufgenommen wurden.
Dennoch ist die Abhängigkeit der Ausgangsspannung von der Phase gut zu erkennen.
Die Abweichungen von der idealen Sinuskurve können auf nicht ideale Bauteile und Rauschen zurückgeführt werden.\\
Die Demodulation des amplitudenmodulierten Signals liefert gute Ergebnisse sowohl beim Ringmodulator und Tiefpass, als auch bei dem Gleichrichter mit Tiefpass.
In den Abbildungen \ref{fig:ringmod} und \ref{fig:gleichrichter} sind die demodulierten Träger und Modulationssignale gut zu erkennen.
Die Hüllenkurven des Modulationssignals stimmen gut mit dem ursprünglichen Modulationssignal überein.


