\section{Discussion}
\label{sec:Diskussion}

The first run of the standard BB84 protocol without Eve shows the expected results of a clean and uninterrupted transmission. The sifted key length of 
$N_{match} = \num{31}$ is $\SI{59.6}{\percent}$ of the total number of transmitted pulses, which is reasonably close to the $\SI{50}{\percent}$ expected for random basis 
choices. The error rate of $N_{error} = \num{0}$ is also expected, as they is no eavesdropper in the system. It also shows that there is no significant noise in the 
system and that the diodes had been calibrated properly. \newline\newline 
With the introduction of Eve and $N _{match} = \num{26}$ matching bases, the error rate of $N_{error} = \num{2}$ and the resulting QBER of $QBER = \SI{7.7}{\percent}$ 
deviate significantly from the expected value of $QBER = \SI{25}{\percent}$. The main systematic reason for this is the low number of transmitted pulses. At 
$N_{total} = \num{52}$, the statistical fluctuations are significant. Upon further investigation, it was found that in this run, Eve's basis matched Alice's basis 
17 out of 26 times, which is $\SI{65.4}{\percent}$. Also, even in the mismatching bases cases,Eve guesses correctly 7 out of 9 cases, which is $\SI{77.8}{\percent}$ 
of the cases. Both of these percentages have an expected value of $\SI{50}{\percent}$, from which each one deviates upwards significantly. This shows that in 
realistic application, the number of signals for key construction must be significantly higher, so that noise error can be distinguished from eavesdropping with 
high confidence. \newline\newline
In the first run of the decoy state method without Eve, the results are again as expected. The sifted key length of $N_{match} = \num{21}$ is reasonably close to
the expected value of $N _{expected} = \num{26}$. The error rate is again zero, which aligns with the first uninterrupted measurement run. \newline\newline
With the introduction of Eve and a sifted key length of $N_{match} = \num{29}$, the error rate of $N_{error} = \num{6}$ and the resulting QBER of $QBER = \SI{20.7}{\percent}$ 
is closer to the expected value of $QBER = \SI{25}{\percent}$. However, the low event number leads to another issue this time. Of the 8 decoy states measured by 
Bob, none of them survived the sifting process, as already mentioned in section \ref{sec:Auswertung}. On the one hand, this limits the non-trivial analysis that could be 
done in this run. On the other hand, this would also a severe security issue in a real-world application, as the detection of decoy states is a crucial aspect 
verifying the presence of an eavesdropper. In combination with the error rate of the first run with Eve included, this could have easily been mistaken for a simple 
noise issue. This further emphasizes the importance of a significantly higher signal number for key construction. 