\section{Setup and Procedure}
\label{sec:Durchführung}

The germanium detector was used to analyze the $\gamma$-spectra of 4 different isotopes. A photo of the actual setup is given in figure~\ref{fig:setup}. 
Each isotope is placed on the mount as seen in the picture. Below, under the visible aluminium shell, the germanium detector is placed. The total setup, 
including source and detector, is placed in a thick lead casing, to prevent $\gamma$-radiation from escaping the system. Underneath the lead casing, connected 
to the germanium detector, is a Dewar vessel filled with liquid nitrogen. This cools the detector as mentioned in setion~\ref{sec:Theorie}. The electronic 
recording and processing units are seen on the right hand side of the picture. Their functions are also explained in section~\ref{sec:Theorie}.

\begin{figure}[h!]
    \centering
    \includegraphics[width=0.8\textwidth]{data/setup.png}
    \caption{Experimental setup for the detection of $\gamma$-radiation with a germanium detector.~\cite{v18_germanium_detector}}
    \label{fig:setup}
\end{figure} 

The 4 different isotopes are $^152$Eu, $^137$Cs, $^133$Ba, and one unknown substance. The probes are placed in the mount one at a time, each for $\SI{90}{\minute}$.
the entire recording happens as an automated process, once it is started via the Computer, which is connected to the processing units. Another recording of 
background radiation is then taken overnight, with no substance in the mount.
