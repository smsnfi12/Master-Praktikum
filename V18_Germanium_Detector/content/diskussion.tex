\section{Discussion}
\label{sec:Diskussion}
In this experiment the properties of a Germanium semiconductor detector were investigated.
The measurement results are generally in good agreement with the theoretical expectations.\\

The Background radiation was measured to be signtificant, that is why it was subtracted from all measured spectra.
As the figure \ref{fig:Energy_calibration} shows, the background spectra has peaks, which means that there are nuclides present in the environment that emit gamma radiation.\\

The energy calibration shows a linear behavior, as expected from theory.
The calibration parameters 
\begin{align*}
    m = (0,119 \pm 0,000)~\text{keV}~,\\
    b = (-0,902 \pm 0,016)~\text{keV}
\end{align*}
have small uncertainties, which indicates a good calibration.
In a perfect case the offset $b$ would be zero, however the small negative value is acceptable, as the measured energies are not located in the low energy range.\\

As expected from theory, the full energy probability decreases with increasing energy.
The measured parameters for the fit function
\begin{align*}
    a &= (29,69 \pm 11,93)~,\\
    b &= (-0,82 \pm 0,08)~
\end{align*}
have relatively large uncertainties, especially for the parameter $a$.
The figure \ref{fig:Q_Eu} shows that some data points deviate significantly from the fit function, which could be a reason for the large uncertainties.\\

The $^{137}$Cs source has one full energy peak at 662 keV as shown in figure \ref{fig:Cs_spectrum}.
The peak was fitted with a Gaussian function to determine the FWHM and FWTM.
The fit parameters have small uncertainties
\begin{align*}
    a & = 3,17 \pm 0,05 , \\
    \mu & = 5560,42 \pm 0,07, \\
    \sigma & = 8,33 \pm 0,10.
\end{align*}
The FWHM and FWTM were calculated in two different ways and both methods yield similar results.\\

From the measured spectrum of the $^{137}$Cs source, the Compton edge and backscatter peak were determined.
The area under the Compton continuum was calculated by integrating the spectrum from the backscatter peak to the Compton edge.
As shown in section \ref{sec:absorption_probability} the theoretical relation between the absorption probabilities at the Compton edge and the full energy peak is
\begin{equation*}
    \frac{P_\text{Compton}}{P_\text{Full}} = 24,89~.
\end{equation*}
Unfortunately the measured area under the Compton continuum and the full energy peak does not yield a similar ratio
\begin{equation*}
    \frac{A_\text{Compton}}{A_\text{Full}} = 2,44~.
\end{equation*}
A possible explaination for this deviation could be that the integration of the Compton continuum was done from the backscatter peak to the Compton edge.
So there is still a significant amount of counts in the Compton continuum below the backscatter peak, which were not considered in the integration.
However, it is not big enough to explain the large deviation from the theoretical value.\\

The measured activity of the $^{133}$Ba source is
\begin{equation*}
    A = (770 \pm 82)~\text{kBq}~.
\end{equation*}
The uncertainty is relatively large, which could be due to the fact that only five peaks were used for the calculation of the activity.
It is also important that the detection probability $Q$ for each peak, was calculated using the fit function from figure \ref{fig:Ba_spectrum}.
The parameter $a$ has a large uncertainty, which could have influenced the calculated activity significantly.\\

The spectrum of the unknown source has several peaks, which were identified and their energies calculated using the energy calibration from section \ref{sec:Energy_calibration}.
The identified nuclides are $^{214}$Pb and $^{214}$Bi.
As they are part of the decay chain of $^{226}$Ra, it can be concluded that the unknown source contains $^{226}$Ra.