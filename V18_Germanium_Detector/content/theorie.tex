\section{Theorie}
\label{sec:Theorie}

\subsection{Interactions of Photons with Matter}

\begin{figure}[h!]
    \centering
    \includegraphics[width=0.8\textwidth]{data/photon_matter_int_sketch.png}
    \caption{Different possible interactions of photons with matter; (a) Photoelectric effect, (b) Compton scattering, (c) Pair production~\cite{kolanoski_wermes_detektoren_2016}}
    \label{fig:int_sketch}
\end{figure}

In the context of this Experiment, there are 3 different mechanisms of Photons interacting with matter. Depicted in figure~\ref{fig:int_sketch} is a 
conceptual visualization of these interactions.\newline 
The Photoelectric effect is a total absorption of a photon by a shell electron of the interacting atom. This process ejects the electron from the atom, 
creating free charge carriers which a detector can collect as a measurable signal. For this effect to occur, the photon's energy must at minimum match the 
binding energy of the electron in the atom. The energy of the ejected electron is therefore 
\begin{equation}
    E_\text{kin} = E_{\gamma} - E_{\text{bind}}\,.
\end{equation}
A process, where only part of the photon's energy is absorbed, is the Compton effect. Here, the photon transfers a portion of its energy to a quasi-free 
electron, depending on the scattering angle $\theta$, as shown in the following equation.
\begin{equation}
    E' = \frac{E}{1+ \frac{E}{m_e c^2}(1-cos(\theta))}\,.
\end{equation}
The Compton edge is defined by the maximum energy transfer at $\theta = \pi$, resulting in the maximum energy of electrons scattered by the Compton effect. 
The probability, that the photon is scattered at a certain solid angle $\Omega$ is described by the differential cross section. For the Compton effect, it is 
given bei the Klein-Nishina equation 
\begin{equation}
    \frac{d\sigma}{d\Omega}
    = 
        \frac{r_e^{\,2}}{2(1 + \varepsilon(1 - \cos\theta))^{2}}
      \left[
          1 + \cos^{2}\theta
          + \frac{\varepsilon^{2}(1 - \cos\theta)^{2}}
                  {1 + \varepsilon(1 - \cos\theta)}
      \right],
\end{equation}\,,
with $\varepsilon = \frac{E}{m_e c^2} $. Figure~\ref{fig:klein} depicts the $theta$-depdency of the cross section. In $phi$ direction, it is constant.
\begin{figure}[h!]
    \centering
    \includegraphics[width=0.8\textwidth]{data/klein_nishina_cs.png}
    \caption{Cross section for Compton scattering~\cite{DelMonte2024}}
    \label{fig:klein}
\end{figure}
Another possible interaction is pair production. In the Coulomb field of the nucleus, a photon can convert into an electron-positron-pair. 
The photon's energy for this interaction to work must be at minimum twice the rest energy of an electron plus the recoil energy produced by the momenta 
of the electron and positron. This momentum can be absorbed by a significantly heavier nucleus at negligable energy cost, in which case the 
energy threshold for the photon is roughly 
\begin{equation}
    E_{\gamma} \approx 2 m_e c^2 \approx \SI{1.022}{\mega\electronvolt}\,.
\end{equation}
Although pair production is technically possible in this experiment, its cross section in germanium in the relevant energy range is significantly lower 
that that of the photoeffect and Compton scattering. This means that its contribution is negligable for our purposes.\newline

A visualization of which processes dominate certain photon energy regions is shown in figure~\ref{fig:att}. Here, the decomposed attenuation coefficient $\mu$ is plotted on the y-axis.
It describes the probability per unit length, that a photon is absorbed or scattered by one of the processes described earlier. It is related to the intensity
of the photon beam via 
\begin{equation}
    I(z) = I_0 e^{-\mu z}\,.
\end{equation}
Neglecting pair production, the Compton effect dominates here for energies larger that $\approx \SI{150}{\kilo\electronvolt}$. Still, since the photoeffect 
scales strongly with the atomic number($\sigma_ph \propto Z^n; n \approx 4-5$), and germanium has a an atomic number of $Z=32$, the photoeffect contributes 
sufficiently in this energy range to allow for well defined full-energy peaks in the $\gamma$-spectrum.
\begin{figure}[h!]
    \centering
    \includegraphics[width=0.6\textwidth]{data/attenuation_coeff.png}
    \caption{Attenuation coefficients of the contributing interactions on an energy spectrum.~\cite{kolanoski_wermes_detektoren_2016}}
    \label{fig:att}
\end{figure} 
\newpage
\subsection{Gamma Spectra}

\begin{figure}[h!]
    \centering
    \includegraphics[width=0.8\textwidth]{data/pr_spectrum.png}
    \caption{Gamma spectrum of the isotope $^{144}\mathrm{Pr}$.~\cite{Geist2024}}
    \label{fig:pr_spec}
\end{figure} 
The decay of a radioactive isotope creates a photon with a well defined energy. Depending on the specific isotope, its decay product can remain in an 
exited state, leading to one or multiple further emissions of photons with characteristic energies. When measured, this results in a spectrum as depicted in 
figure~\ref{fig:spec}. The well defined peaks in the count rates at certain energies can be used to identify the emitting isotope. 
