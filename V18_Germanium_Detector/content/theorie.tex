\section{Theory}
\label{sec:Theorie}

\subsection{Interactions of Photons with Matter}

\begin{figure}[h!]
    \centering
    \includegraphics[width=0.8\textwidth]{data/photon_matter_int_sketch.png}
    \caption{Different possible interactions of photons with matter; (a) Photoelectric effect, (b) Compton scattering, (c) Pair production~\cite{kolanoski_wermes_detektoren_2016}}
    \label{fig:int_sketch}
\end{figure}

In the context of this Experiment, there are 3 different mechanisms of Photons interacting with matter. Depicted in figure~\ref{fig:int_sketch} is a 
conceptual visualization of these interactions.\newline 
The Photoelectric effect is a total absorption of a photon by a shell electron of the interacting atom. This process ejects the electron from the atom, 
creating free charge carriers which a detector can collect as a measurable signal. For this effect to occur, the photon's energy must at minimum match the 
binding energy of the electron in the atom. The energy of the ejected electron is therefore 
\begin{equation}
    E_\text{kin} = E_{\gamma} - E_{\text{bind}}\,.
\end{equation}
A process, where only part of the photon's energy is absorbed, is the Compton effect. Here, the photon transfers a portion of its energy to a quasi-free 
electron, depending on the scattering angle $\theta$, as shown in the following equation.
\begin{equation}
    E' = \frac{E}{1+ \frac{E}{m_e c^2}(1-cos(\theta))}\,.
\end{equation}
The Compton edge is defined by the maximum energy transfer at $\theta = \pi$, resulting in the maximum energy of electrons scattered by the Compton effect. 
The probability, that the photon is scattered at a certain solid angle $\Omega$ is described by the differential cross section. For the Compton effect, it is 
given bei the Klein-Nishina equation 
\begin{equation}
    \frac{d\sigma}{d\Omega}
    = 
        \frac{r_e^{\,2}}{2(1 + \varepsilon(1 - \cos\theta))^{2}}
      \left[
          1 + \cos^{2}\theta
          + \frac{\varepsilon^{2}(1 - \cos\theta)^{2}}
                  {1 + \varepsilon(1 - \cos\theta)}
      \right],
\end{equation}\,,
with $\varepsilon = \frac{E}{m_e c^2} $. Figure~\ref{fig:klein} depicts the $theta$-depdency of the cross section. In $phi$ direction, it is constant.
\begin{figure}[h!]
    \centering
    \includegraphics[width=0.8\textwidth]{data/klein_nishina_cs.png}
    \caption{Cross section for Compton scattering~\cite{DelMonte2024}}
    \label{fig:klein}
\end{figure}
Another possible interaction is pair production. In the Coulomb field of the nucleus, a photon can convert into an electron-positron-pair. 
The photon's energy for this interaction to work must be at minimum twice the rest energy of an electron plus the recoil energy produced by the momenta 
of the electron and positron. This momentum can be absorbed by a significantly heavier nucleus at negligable energy cost, in which case the 
energy threshold for the photon is roughly 
\begin{equation}
    E_{\gamma} \approx 2 m_e c^2 \approx \SI{1.022}{\mega\electronvolt}\,.
\end{equation}
Although pair production is technically possible in this experiment, its cross section in germanium in the relevant energy range is significantly lower 
that that of the photoeffect and Compton scattering. This means that its contribution is negligable for our purposes.\newline

A visualization of which processes dominate certain photon energy regions is shown in figure~\ref{fig:att}. Here, the decomposed attenuation coefficient $\mu$ is plotted on the y-axis.
It describes the probability per unit length, that a photon is absorbed or scattered by one of the processes described earlier. It is related to the intensity
of the photon beam via 
\begin{equation}
    I(z) = I_0 e^{-\mu z}\,.
\end{equation}
Neglecting pair production, the Compton effect dominates here for energies larger that $\approx \SI{150}{\kilo\electronvolt}$. Still, since the photoeffect 
scales strongly with the atomic number($\sigma_ph \propto Z^n; n \approx 4-5$), and germanium has an atomic number of $Z=32$, the photoeffect contributes 
sufficiently in this energy range to allow for well defined full-energy peaks in the $\gamma$-spectrum.
\begin{figure}[h!]
    \centering
    \includegraphics[width=0.6\textwidth]{data/attenuation_coeff.png}
    \caption{Attenuation coefficients of the contributing interactions on an energy spectrum.~\cite{kolanoski_wermes_detektoren_2016}}
    \label{fig:att}
\end{figure} 
\newpage
\subsection{Gamma Spectra}

\begin{figure}[h!]
    \centering
    \includegraphics[width=0.8\textwidth]{data/pr_spectrum.png}
    \caption{Gamma spectrum of the isotope $^{144}\mathrm{Pr}$.~\cite{Geist2024}}
    \label{fig:pr_spec}
\end{figure} 
The decay of a radioactive isotope creates a photon with a well defined energy. Depending on the specific isotope, its decay product can remain in an 
exited state, leading to one or multiple further emissions of photons with characteristic energies. When measured, this results in a spectrum as depicted in 
figure~\ref{fig:pr_spec}. The well defined peaks in the count rates at certain energies can be used to identify the emitting isotope. \newline
Some isotopes have a monochromatic spectrum, since emission of a singular photon leaves the daughter nucleus in its ground state. Such an isotope allows for 
a more detailed analysis of its spectrum. An element with this characteristic is $^{137}\mathrm{Cs}$, for which an energy spectrum is depicted in 
figure~\ref{fig:cs_spec}.
\begin{figure}[h!]
    \centering
    \includegraphics[width=0.8\textwidth]{data/cs_spectrum.png}
    \caption{Gamma spectrum of the isotope $^{137}\mathrm{Cs}$.~\cite{Geist2024}}
    \label{fig:cs_spec}
\end{figure}
The 'full energy peak' describes the events, where the emitted photons where directly absorbed by the detector via the photoeffect with their original energy. 
This is the peak of which multiple are seen in spectrums like~\ref{fig:pr_spec}. The counts at higher energies can be explained by a 'pulse pile-up' where 
the detector is recording multiple photons at once, leading to larger energy measurements. When the photons interacts with the detector material via 
Compton scattering instead, only part of its energy is absorbed by an electron. the Compton edge described the maximum energy transfer from a photon to an 
electron via the Compton effect. The 'backscattering' peak is explained by a photon being scattered outside of the detector at a large enough angle to get 
back inside the detector with a lower energy, which is then absorbed by the photoeffect. The energy region between the full energy peak and the Compton edge 
is where multiple Compton scattering events are detected simultaneously.\newline\newline
Given a system with probabilistic emissions, for which the detection probability is constant over time, the statistics can be described via a Poisson distribution. 
The uncertainty is therefore given by $\Delta N =\sqrt{N}$. The probability that a photon interacts at least once in a detector of thickness $d$, with a total 
attenuation coefficient $\mu$ is then given by 
\begin{equation}
    P(d) = 1 - e^{-\mu d}\,.
\end{equation} 
The full-energy detection probability is the probability, that a given photon entering the detector, deposits its total energy, contributing to the mentioned 
full-energy peak. This can be experimentally determined and is calculated via 
\begin{equation}
    Q = \frac{N_{\text{Peak}}}{A\, t\, \omega_\gamma} \, \frac{4\pi}{\Omega} \, .
\end{equation}

Here, $A$ is the source activity, $\omega_\gamma$ is the branching ratio of the $\gamma$-line, $t$ is the measuring time, and $\Omega$ is the solid angle, the 
detector encompasses. $\Omega$ can be be calculated knowing the parameters of the experiment's setup. In this case, it is given by 
\begin{equation}
    \Omega = 2\pi \Bigl( 1- \frac{a}{\sqrt{a^2 + r^2}}\Bigr)\,.
\end{equation}
Here, $a= \SI{7.02}{\centi\meter}$ is the distance from the source to the detector and $r=d/2=\SI{2.25}{\cm}$ is the radius of the detector, assuming 
a circular construction. The effective solid angle is therefore $\frac{\Omega}{4\pi} = 0.016$

\subsection{Detection Mechanism}

\subsubsection{Semiconductors}

A semiconductor is characterized by its band structure. The key property is the separation between the valence band and the conduction band. In a metal, these 
bands overlap, while in an insolater, the energy gap is large. In a semiconductor, the separation between the bands is small enough, for electrons of the 
valence band, to get exited into the conduction band by thermal exitation of external energy deposition (the latter in this experiment). An exitation leaves 
a positively charged electron 'hole' in the valence band. The resulting electron-hole pair then act as mobile charge carriers. When a photon deposits its energy
in the semiconductor, it creates a number of such pairs, which is proportional to the deposited energy via $ N_{eh} = \frac{E_{depo}}{E_{ex}}$, where $E_{ex}$
is the mean energy required for an exitation. 

\subsubsection{The Germanium Detector}

Germanium is a well suited semiconductor for $\gamma$-radiation detection for multiple reasons. Its relatively high atomic number $Z=32$ makes for a large 
photoelectric cross section, which leads to well defined full-energy peaks, as discussed earlier. Also, its small band width allows for low energy deposition 
to be detected, which makes for high energy resolution. The small band width however allows also for thermal exitations at room temperature, necessatating 
liquid-nitrogen cooling of the detector during measurement. \newline 
When using germanium as a detector, it is important to reduce naturally free electrons in the conduction band as well as holes in the valence band, which can 
exist without exitation. For this, two heavily doped regions n$^+$ and p$^+$ are created. The n$^+$ layer is created by lithium diffusion, providing excess 
electrons and a conductive contact, while the p$^+$ layer is realized as an evaporated gold film, which also seres as a conductive contact, and provides a 
region with high hole concentration. With this construct, additionally applying a large enough reverse bias voltage to the contacts creates a fully depleted 
region in the semiconductor, removing any free charge carriers from the system. If a photon interacts with this region, creating an electron-hole pair, the 
electrons move towards the n$^+$ region, while the holes move towards the p$^+$ region. This creates a transient current, which can be measured and is 
proportional to number of e-h pair and thus proportional to the deposited energy. 

\subsection{Signal Collection and Processing}

As explained in the previous section, a photon depositing its energy creates a measurable current pulse. The processing of this signal happens in multiple 
stages. First, a charge-sensitive preamplifier integrates the induced current on a feedback capacitor, producing a voltage step proportional to the total 
collected charge. The output is further  processed by a shaping amplifier, filtering noise and transforming the step function signal into a Gaussian 
standardized pulse. This improves energy resolution, ensuring distinguishing of pulses corresponding to different energies. To further supress noise, a 
discriminator is used to reject pulses below a certain threshold. The shaped pulses then enter a multichannel analyzer, which converts the analogue
into a digital signal. A pulse amplitude is assigned to a corresponding discrete channel number. The subsequent output is the final $\gamma$-spectrum of 
counts against channel number, where each channel number corresponds to a certain energy. A sketch of the circuit as explained, is depicted in 
figure~\ref{fig:circuit}.
\begin{figure}[h!]
    \centering
    \includegraphics[width=0.8\textwidth]{data/circuit.png}
    \caption{Circuit for the Signal Processing in a Germanium detector.~\cite{kolanoski_wermes_detektoren_2016}}
    \label{fig:circuit}
\end{figure} 


