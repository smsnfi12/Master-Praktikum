\section{Analysis}
\label{sec:Auswertung}
Alle Berechnungen werden mit dem Programm \glqq Numpy" \cite{numpy}, die Unsicherheiten mit dem Modul \glqq Uncertainties" \cite{uncertainties}, die Ausgleichsrechnungen mit dem Modul \glqq Scipy" \cite{scipy} durchgeführt und die grafischen Darstellungen über das Modul \glqq Matplotlib" \cite{matplotlib} erstellt.

\subsection{$^{152}$Eu source}
The figure \ref{fig:Energy_calibration} shows the spectrum of the $^{152}$Eu source and the background normalized by the measurement time. 
The measurement time for the $^{152}$Eu source is $t_{\text{Eu}} = 6266$ s and for the background $t_{\text{bg}} = 75671$ s.
\begin{figure}[H]
    \centering
    \includegraphics[width=0.8\textwidth]{plots/Eu_spectrum.pdf}
    \caption{Spectrum of the $^{152}$Eu source and the background normed by the time.}
    \label{fig:Energy_calibration}
\end{figure}

\subsubsection{Energy Calibration}
To calibrate the energy scale of germanium detector, the background is subtracted from the measured spectrum of the $^{152}$Eu source.
The resulting spectrum with the identified peaks is shown in figure \ref{fig:Energy_calibration_fit}.
The peaks represent the full energy absorption of gamma quanta in the detector.\\


%The parameters of the Gaussian fits are used to determine the channel numbers corresponding to the peak energies listed
\begin{figure}[H]
    \centering
    \includegraphics[width=0.8\textwidth]{plots/Eu_peaks.pdf}
    \caption{Identified peaks of the $^{152}$Eu source.}
    \label{fig:Energy_calibration_fit}
\end{figure}

On the peaks is performed a Gaussian fit as shown in figures \ref{fig:Peak1_fit} and \ref{fig:Peak7_fit} for peak 1 and peak 7, respectively.
It can be seen that peak 1 fits very well, while peak 7 has deviations from the Gaussian shape.
Because of the deviations the peak 7 is not considered for the energy calibration.\\

\begin{figure}[H]
    \centering
    \begin{minipage}{0.45\textwidth}
        \centering
        \includegraphics[width=0.9\textwidth]{plots/Eu_peak1.pdf}
        \caption{Fit of Peak 1.}
        \label{fig:Peak1_fit}
    \end{minipage}
    \hfill
    \begin{minipage}{0.45\textwidth}
        \centering
        \includegraphics[width=0.9\textwidth]{plots/Eu_peak7.pdf}
        \caption{Fit of Peak 7.}
        \label{fig:Peak7_fit}
    \end{minipage}
\end{figure}
For all identified peaks a gaussian fit 
\begin{equation*}
    f(x) = a \cdot \exp\left(-\frac{(x - \mu)^2}{2 \sigma^2}\right) + b
\end{equation*}
is performed. The peak areas are determined by integrating the Gaussian functions.
The peaks in the figures above show the normed counts per second for each channel. 
To determine the total counts under the peak, the area is multiplied by the measurement time of the $^{152}$Eu source $t_{\text{Eu}} = 6266$ s.
The peaks and their corresponding areas are listed in table \ref{tab:livetimes}.\\
To calibrate the energy scale with the $^{152}$Eu source, the known energies of the peaks must be assigned to the corresponding channel numbers.
The energy values and their probabilities are taken from \cite{eu152}.
The table \ref{tab:livetimes} lists the energies with the greatest probability of emission for $^{152}$Eu, their intensities, the corresponding channel numbers and the determined peak areas.\\

\begin{table}
    \centering
    \caption{Energies with greatest probability of Emission for $^{152}$Eu}
    \label{tab:livetimes}
    \begin{tabular}{c c |c c}
        \toprule
        Energy $E_{\text{lit}}$ [keV] & Probability $P_{\text{lit.}}$ [$\%$]  & Channel &  Peak Area $a$\\
        \midrule
        121,78 & 28,41 \pm 0,13 & 1029 & 15390 \pm 139\\
        244,70 & 7,55 \pm 0,04 & 2061 & 2752 \pm 77\\
        344,28 & 26,60 \pm 0,12 & 2897 & 6359 \pm 105\\
        411,12 & 2,24 \pm 0,10 & 3458 & 421 \pm 40\\
        443,97 & 2,80 \pm 0,02 & 3733 & 535 \pm 35\\
        778,90 & 12,97 \pm 0,06 & 6543 & 1202 \pm 68\\
        % % 867,37 & 4,24 \pm 0,23 &  & \\
        % % 964,09 & 14,50 \pm 0,06 &  & \\
        % % 1085,84 & 10,13 \pm 0,06 &  & \\
        % % 1112,08 & 13,41 \pm 0,06 &  & \\
        % % 1408,01 & 20,85 \pm 0,08 &  & \\
        \bottomrule
    \end{tabular}
\end{table}
The calibration is performed by a linear fit 
\begin{equation*}
    E_{\text{lit}} = m \cdot K + b
\end{equation*}
of the energy values $E_{\text{lit}}$ over the corresponding channel numbers $K$ as shown in figure \ref{fig:Energy_calibration_fit_linear}.

\begin{figure}[H]
    \centering
    \includegraphics[width=0.8\textwidth]{plots/Energy_calibration.pdf}
    \caption{Energy Calibration of the Germanium Detector with the $^{152}$Eu source.}
    \label{fig:Energy_calibration_fit_linear}
\end{figure}
The resulting fit parameters are:
\begin{align}
    m & = (0,119 \pm 0,000 )~\text{keV},\\
    b & = (-0,902 \pm 0,016 )~\text{keV}.
    \label{eq:callibration}
\end{align}
The uncertainty of the parameter $m$ is $\Delta m = 4,36 \times 10^{-6}$ keV.

\subsubsection{Determination of the Full-Energy Probability}
The full energy detection probability $Q$ is determined by the relation \ref{eq:Q}:
\begin{equation}
    Q = \frac{N}{A P t} \cdot \frac{4\pi}{\Omega},
\end{equation}
where $N$ is the peak area, $A$ the activity of the source at the time of measurement, $P$ the emission probability, $t$ the measurement time and $\frac{\Omega}{4\pi}$ the solid angle fraction covered by the detector.
The angle fraction $\frac{\Omega}{4\pi}$ is calculated in the section \ref{sec:gamma_spectra}.
The activity $A$ of the source at the time of the measurement is calculated as follows:
\begin{equation*}
    A = A_0 \cdot \exp\left(-\frac{\ln(2)}{T_{1/2}} t\right),
\end{equation*}
where $A_0 = (4130 \pm 60)$ Bq is the activity at the reference time (01.01.2000), $T_{1/2} = 4944$ days the half-life of $^{152}$Eu, and $t = 9171$ days the time difference between the reference time and the measurement time.
The activity at the time of measurement is calculated to be
\begin{equation*}
    A = (1142 \pm 17)~\text{Bq}.
\end{equation*}
The calculated values for $Q$ are listed in table \ref{tab:Q_values}.
\begin{table}
    \centering
    \caption{Full Energy detection probability $Q$ for the identified peaks of the $^{152}$Eu source.}
    \label{tab:Q_values}
    \begin{tabular}{c c }
        \toprule
        Energy [keV] & $Q$ [$\%$]\\
        \midrule
        121,78 & 0,563 \pm 0,009\\
        244,70 & 0,378 \pm 0,012\\
        344,28 & 0,248 \pm 0,005\\
        411,12 & 0,195 \pm 0,020\\
        443,97 & 0,198 \pm 0,006 \\
        778,90 & 0,096 \pm 0,006 \\
        % % 867,37 & 4,24 \pm 0,23 &  & \\
        % % 964,09 & 14,50 \pm 0,06 &  & \\
        % % 1085,84 & 10,13 \pm 0,06 &  & \\
        % % 1112,08 & 13,41 \pm 0,06 &  & \\
        % % 1408,01 & 20,85 \pm 0,08 &  & \\
        \bottomrule
    \end{tabular}
\end{table}
Considering the determined values for $Q$ and the corresponding energies $E$, a fit of the potential function
\begin{equation*}
    Q(E) = a \cdot E^{\prime ~ b }
\end{equation*}
is performed as shown in figure \ref{fig:Q_Eu}, where $E^{\prime}$ is the energy normalized to 1 keV.

\begin{figure}[H]
    \centering
    \includegraphics[width=0.8\textwidth]{plots/Q_Eu.pdf}
    \caption{Full Energy detection probability $Q$ depending on the energy $E$ for the $^{152}$Eu source.}
    \label{fig:Q_Eu}
\end{figure}
The resulting fit parameters are:
\begin{align}
    a & = 29,69 \pm 11,93,\\
    b & = -0,82 \pm 0,08.
\end{align}
So the full energy detection probability can be described as
\begin{equation*}
    Q(E) = (29,69 \pm 11,93) \cdot E^{\prime ~ -0,82 \pm 0,08}.
\end{equation*}

\subsection{$^{137}$Cs source}
The monochromatic spectrum of the $^{137}$Cs source is shown in figure \ref{fig:Cs_spectrum}, where the background is subtracted and the counts are normalized by the measurement time $t = 6382$ s.
It can be seen that the spectrum consists of a prominent full energy peak, a Compton continuum and a backscatter peak.
\begin{figure}[H]
    \centering
    \includegraphics[width=0.8\textwidth]{plots/Cs_spectrum.pdf}
    \caption{Spectrum of the $^{137}$Cs source and the background normed by the time.}
    \label{fig:Cs_spectrum}
\end{figure}
\subsubsection{Full Energy Peak Analysis}
The full energy peak of the $^{137}$Cs source at $662,69$ keV is fitted with a Gaussian function
\begin{equation*}
    f(x) = \frac{a}{\sqrt{2 \pi \sigma ^2}} \cdot \exp\left(-\frac{(x - \mu)^2}{2 \sigma^2}\right) + b
\end{equation*}
as shown in figure \ref{fig:Cs_peak}.
\begin{figure}[H]
    \centering
    \includegraphics[width=0.8\textwidth]{plots/Cs_peak_gauss.pdf}
    \caption{Fit of the Cs Peak with marked FWHM and FWTM.}
    \label{fig:Cs_peak}
\end{figure} 
The corresponding Gaussian parameters are
\begin{align*}
    a & = 3,17 \pm 0,05 , \\
    \mu & = 5560,42 \pm 0,07, \\
    \sigma & = 8,33 \pm 0,10.
\end{align*}
The figure \ref{fig:Cs_peak} also shows the Full Width at Half Maximum (FWHM) and the Full Width at Tenth Maximum (FWTM) of the peak.
FWHM and FWTM can be determined by the data directly or by the Gaussian fit parameters.\\
The maximal count rate of the measured data is $N_\text{max} = 1003$.
The FWHM and FWTM count rates are accordingly determined to be
\begin{align*}
    N_\text{FWHM} & =\frac{N_\text{max}}{2} = 501,5~,\\
    N_{\text{FWHM}} & =\frac{N_\text{max}}{10} = 100,3~.
\end{align*}
The corresponding channel numbers can be looked up in the data.
The width of the peak in channels is determined by the difference of the two channel numbers at the FWHM and FWTM count rates
\begin{equation*}
    \text{Width} = K(N_\text{right}) - K(N_\text{left}).
\end{equation*}
So the widths determined directly from the data are
\begin{align*}
    FWHM & = 18 ~,\\
    FWTM & = 35~.
\end{align*}
%The channel number uncertainties are estimated to be $\approx 10$ channels.\\
The maximal count rate of the Gaussian fit is at 
\begin{equation*}
    N_\text{max} = f(\mu) = \frac{a}{\sqrt{2 \pi \sigma ^2}} = 1001,90
\end{equation*}
The resulting FWHM and FWTM count rates are calculated analogously to the data directly
\begin{align*}
    N_{\text{FWHM}} & = 500,95~,\\
    N_{\text{FWHM}} & = 100,19~.    
\end{align*}
The corresponding widths in channels are determined to be
\begin{align*}
    FWHM & = 19,63~,\\
    FWTM & = 35,77~.
\end{align*}
To describe the widths in energy, the calibration parameters from the section \ref{sec:Auswertung} are used to convert the channel numbers to energies.
The resulting widths in energy are
\begin{align*}
    FWHM & = 2,34~ \text{keV},\\
    FWTM & = 4,26~ \text{keV}.
\end{align*}
The ratio of FWTM to FWHM is
\begin{equation*}
    \frac{FWTM}{FWHM} = 1,82~.
\end{equation*}
The $^{137}$Cs peak area is determined by integrating the Gaussian function
\begin{align*}
    N_{\text{peak}} & =  20232 \pm 398 ~\text{counts}.
\end{align*}
The corresponding energy of the peak is
\begin{equation}
    E_{\text{Cs}} = 662,69 ~\text{keV}.
    \label{eq:Cs_energy}
\end{equation}

\subsubsection{Compton continuum}
The Compton edge and the backscatter peak can be obtained from the equation \ref{eq:Compton}.
The Compton edge corresponds to the maximum energy transfer at a scattering angle of $\theta = \pi$
\begin{equation*}
    E_{\text{Compton-Edge}} = E - E'(\theta = \pi) = E \cdot \frac{\epsilon (1-\cos{\pi})}{1+\epsilon (1-\cos{\pi})} = E \cdot \frac{2 \epsilon}{1 + 2 \epsilon} ~,
\end{equation*}
where $\epsilon = \frac{E}{m_e c^2}$.
The backscatter peak corresponds to the photons that are scattered at $\theta = \pi$ outside the detector and then are scattered into the detector at $\theta = 0$
\begin{equation*}
    E_{\text{Backscatter}} = E'(\theta = \pi) = \frac{E}{1 + 2 \epsilon} ~.
\end{equation*}
From the measured energy peak of the $^{137}$Cs source (equation \ref{eq:Cs_energy}) the Compton edge and the backscatter peak can be calculated to be
\begin{align*}
    E_{\text{Compton-Edge}} & = 478,28 ~\text{keV}~,\\
    E_{\text{Backscatter}} & = 184,40 ~\text{keV}~.
\end{align*}
The Compton edge and the backscatter peak can also be determined experimentally from the measured spectrum data.
The Compton edge is located at the point with the channel $K=3912$ and the backscatter peak at the channel $K=1618$.
The channel uncertainties are estimated to be approximately $10$ channels.
The channel numbers are converted to energies using the calibration parameters from equation \ref{eq:callibration}
\begin{align*}
    E_{\text{Compton-Edge}} & = 466,18 \pm 0,12 \text{keV}\\
    E_{\text{Backscatter}} & = 192,81 \pm 0,12 \text{keV}
\end{align*}
The determined Compton continuum with the backscatter peak are shown in figure \ref{fig:Cs_peak_data}.

\begin{figure}[H]
    \centering
    \includegraphics[width=0.8\textwidth]{plots/Cs_compton.pdf}
    \caption{Compton continuum and backscatter peak of the $^{137}$Cs source.}
    \label{fig:Cs_peak_data}
\end{figure}

To determine the area under the Compton continuum, a linear fit
\begin{equation*}
    N(K) = m \cdot K + b
\end{equation*}
is performed from the channel number of the Compton edge $N_\text{Compton} = 3912$ to the channel number of the backscatter peak $N_\text{Backscatter} = 1618$ as shown in figure \ref{fig:Cs_compton_fit}.

\begin{figure}
    \centering
    \includegraphics[width=0.8\textwidth]{plots/Cs_linear.pdf}
    \caption{Linear fit of the Compton continuum of the $^{137}$Cs source.}
    \label{fig:Cs_compton_fit}
\end{figure}
The resulting fit parameters are
\begin{align*}
    m & = (-1,66 \pm 0,33) \cdot 10^{-7} \text{1/s}~,\\
    b & = (3,84 \pm 0,09) \cdot 10^{-3} \text{1/s}~.
\end{align*}
So the area under the Compton continuum is determined to be
\begin{align*}
    N_{\text{Compton}} & =  8,33 ~\text{counts/s}.
\end{align*}
The total number of counts under the Compton continuum during the measurement time of $t = 6382$ s is
\begin{equation*}
    N_{\text{Compton}} = 53173 ~\text{counts}.
\end{equation*}
The ratio of the peak area to the Compton continuum area is
\begin{equation*}
    \frac{N_{\text{Compton}}}{N_{\text{Full Energy Peak}}} = \frac{8,33~\text{s}^{-1}}{3,17~\text{s}^{-1}} = 2,62 ~.
\end{equation*}


\subsubsection{Absorption Probability}
\label{sec:absorption_probability}
The absorption probability can be determined by the relation \eqref{eq:absorption Probability}, where $d = 3,9$ cm is the thickness of the germanium detector and $\mu$ the attenuation coefficient.
The attenuation coefficients for the energies of the full energy peak and the Compton edge are
\begin{align*}
    \mu_{\text{Compton}} & = 0,37 ~\text{cm}^{-1}~,\\
    \mu_{\text{Full Energy}} & = 0,008 ~\text{cm}^{-1}~.
\end{align*}
The resulting absorption probabilities are
\begin{align*}
    P_{\text{Compton}} & = 76,38 \%~,\\
    P_{\text{Full Energy}} & = 3,07 \%~.
\end{align*}
The relation between the absorption probability of the Compton edge and the full energy peak is
\begin{equation*}
    \frac{P_{\text{Compton}}}{P_{\text{Full Energy}}} = 24,89 ~,
\end{equation*}
which means that the area under the Compton continuum is approximately 25 times larger than the area under the full energy peak.


\subsection{$^{133}$Ba source}
The spectrum of the $^{133}$Ba source with the background normalized by the measurement time is shown in figure \ref{fig:Ba_spectrum}.
The measurement time for the $^{133}$Ba source is $t_{\text{Ba}} = 4455$ s.
\begin{figure}[H]
    \centering
    \includegraphics[width=0.8\textwidth]{plots/Ba_spectrum.pdf}
    \caption{Spectrum of the $^{133}$Ba source normed by the time and the background.}
    \label{fig:Ba_spectrum}
\end{figure}
\subsubsection{Activity Determination}
The activity $A$ of the source $^{133}$Ba can be determined with the relation \ref{eq:Q} rearranged to
\begin{equation*}
    A = \frac{N}{Q P t} \cdot \frac{4\pi}{\Omega}~.
\end{equation*}
The determined values for $A$ for the identified peaks are listed in table \ref{tab:Ba_peaks}.
The energies and their probabilities are taken from \cite{eu152}.

\begin{table}
    \centering
    \caption{Energies with greatest probability of Emission for $^{133}$Ba}
    \label{tab:Ba_peaks}
    \begin{tabular}{c c c c c c}
        \toprule
        Channel & Energy [keV] & Area $N/t$ [1/s] & $Q$ & $P$ [$\%$] & Aktivität A [Bq]\\
        \midrule
        688 & 80,99 \pm 0,11 & 1,78 & 0,81 & 33,31 & 489,38\\
        2327 & 276,40 \pm 0,12 & 0,28 & 0,30 & 7,13 & 993,85\\
        2550 & 302,85 \pm 0,05 & 0,56 & 0,27 & 18,31 & 828,79\\
        2997 & 356,01 \pm 0,07 & 1,59 & 0,24 & 62,05 & 790,78\\
        3226 & 383,85 \pm 0,12 & 0,20 & 0,23 & 8,94 & 745,30\\
        \bottomrule
    \end{tabular}
\end{table}
The mean activity of the $^{133}$Ba source is determined to be
\begin{equation*}
    \langle A \rangle = (770 \pm 82)~\text{Bq}.
\end{equation*}

\subsection{Nuclide Identification}
The spectrum of the unknown source with the background normalized by the measurement time is shown in figure \ref{fig:Ba_spectrum}.
To find out which nuclides are present in the unknown source, the identified peaks are compared with known energies from \cite{eu152}.
The identified peaks and their corresponding nuclides are listed in table \ref{tab:unknown_peaks}.
\begin{figure}[H]
    \centering
    \includegraphics[width=0.8\textwidth]{plots/Unknown_spectrum.pdf}
    \caption{Identified peaks of the $^{133}$Ba source.}
    \label{fig:Ba_peaks}
\end{figure}

\begin{table}
    \centering
    \caption{Identified peaks of the unknown spectrum.}
    \label{tab:unknown_peaks}
    \begin{tabular}{c c c}
        \toprule
        Channel & Energy [keV] & Nuclide\\
        \midrule
        654 & 77 & $/$ \\
        783 & 92 & $^{234}$Th \\
        1570 & 186 & $^{226}$Ra \\
        2038 & 242 & $^{214}$Pb \\
        2486 & 295 & $^{214}$Pb \\
        2961 & 352 & $^{214}$Pb \\
        5120 & 609 & $^{214}$Bi \\
        5598 & 666 & $^{214}$Bi \\
        6457 & 769 & $^{214}$Bi \\
        6768 & 806 & $^{214}$Bi \\
        7845 & 934 & $^{214}$Bi \\
        \bottomrule
    \end{tabular} 
\end{table}
It can be seen, that the unknown source contains $^{214}$Pb, which decays into $^{214}$Bi.
Both of the nuclides were identified in the unknown source spectrum.
$^{214}$Pb and $^{214}$Bi are part of the decay chain of $^{226}$Ra.
