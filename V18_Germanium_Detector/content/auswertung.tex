\section{Auswertung}
\label{sec:Auswertung}
Alle Berechnungen werden mit dem Programm \glqq Numpy" \cite{numpy}, die Unsicherheiten mit dem Modul \glqq Uncertainties" \cite{uncertainties}, die Ausgleichsrechnungen mit dem Modul \glqq Scipy" \cite{scipy} durchgeführt und die grafischen Darstellungen über das Modul \glqq Matplotlib" \cite{matplotlib} erstellt.

\subsection{Energy Calibration and determination of the full-energy probability}


\begin{figure}[H]
    \centering
    \includegraphics[width=0.8\textwidth]{plots/Eu_spectrum.pdf}
    \caption{Spectrum of the $^{152}$Eu source and the background normed by the time.}
    \label{fig:Energy_calibration}
\end{figure}

\begin{figure}[H]
    \centering
    \includegraphics[width=0.8\textwidth]{plots/Eu_peaks.pdf}
    \caption{Identified peaks of the $^{152}$Eu source.}
    \label{fig:Energy_calibration_fit}
\end{figure}

\begin{figure}[H]
    \centering
    \begin{minipage}{0.45\textwidth}
        \centering
        \includegraphics[width=0.9\textwidth]{plots/Eu_peak1.pdf}
        \caption{Fit of Peak 1.}
        \label{fig:Peak1_fit}
    \end{minipage}
    \hfill
    \begin{minipage}{0.45\textwidth}
        \centering
        \includegraphics[width=0.9\textwidth]{plots/Eu_peak7.pdf}
        \caption{Fit of Peak 7.}
        \label{fig:Peak7_fit}
    \end{minipage}
\end{figure}



\begin{table}
    \centering
    \caption{Energies wirh greatest probability of Emission for $^{152}$Eu}
    \label{tab:livetimes}
    \begin{tabular}{c c |c c}
        \toprule
        Energy [keV] & Intensity [$\%$]  & Channel & Linieninhalt\\
        \midrule
        121,78 & 28,41 \pm 0,13 & 1029 & 15390 \pm 139\\
        244,70 & 7,55 \pm 0,04 & 2061 & 2752 \pm 77\\
        344,28 & 26,60 \pm 0,12 & 2897 & 6359 \pm 105\\
        411,12 & 2,24 \pm 0,10 & 3458 & 421 \pm 40\\
        443,97 & 2,80 \pm 0,02 & 3733 & 535 \pm 35\\
        778,90 & 12,97 \pm 0,06 & 6543 & 1202 \pm 68\\
        % % 867,37 & 4,24 \pm 0,23 &  & \\
        % % 964,09 & 14,50 \pm 0,06 &  & \\
        % % 1085,84 & 10,13 \pm 0,06 &  & \\
        % % 1112,08 & 13,41 \pm 0,06 &  & \\
        % % 1408,01 & 20,85 \pm 0,08 &  & \\
        \bottomrule
    \end{tabular}
\end{table}


\begin{figure}[H]
    \centering
    \includegraphics[width=0.8\textwidth]{plots/Energy_calibration.pdf}
    \caption{Energy Calibration of the Germanium Detector with the $^{152}$Eu source.}
    \label{fig:Energy_calibration_fit}
\end{figure}

\begin{align*}
    m & = 0,119 \pm 0,000 \text[keV],\\
    b & = -0,902 \pm 0,016 .
\end{align*}

$\delta m = 4,36 \times 10^{-6}$


\begin{figure}[H]
    \centering
    \includegraphics[width=0.8\textwidth]{plots/Q_Eu.pdf}
    \caption{Full Energy detection probability $Q$ depending on the energy $E$ for the $^{152}$Eu source.}
    \label{fig:Q_Eu}
\end{figure}


\begin{equation}
    Q = \frac{Z}{AWt} \cdot \frac{4\pi}{\Omega}
\end{equation}

\begin{equation}
    Q(E) = a \cdot E^{b}
\end{equation}

\begin{align}
    a & = 29,69 \pm 11,93,\\
    b & = -0,82 \pm 0,08.
\end{align}


\subsection{Monochromatic gamma spectrum}

\begin{figure}[H]
    \centering
    \includegraphics[width=0.8\textwidth]{plots/Cs_spectrum.pdf}
    \caption{Spectrum of the $^{137}$Cs source and the background normed by the time.}
    \label{fig:Cs_spectrum}
\end{figure}

Gaussian parameters:
\begin{align*}
    a & = 3,17 \pm 0,05\\
    \mu & = 5560,42 \pm 0,07\\
    \sigma & = 8,33 \pm 0,10
\end{align*}

Datenpatameter:
\begin{align*}
    N_{\text{max}} & = 1003\\
    N_\text{FWHM} & = 501,5\\
    N_{\text{FWHM}} & = 100,3
\end{align*}
Breite:
\begin{align*}
    FWHM & = 18\\
    FWTM & = 35
\end{align*}

Kurvenparameter:
\begin{align*}
    N_{\text{max}} & = 1001,90\\
    N_{\text{FWHM}} & = 500,95\\
    N_{\text{FWHM}} & = 100,19
\end{align*}
Breite:
\begin{align*}
    FWHM & = 19,63\\
    FWTM & = 35,77
\end{align*}

Theorire:
\begin{align*}
    N_{\text{FWHM}} & = 2 \sqrt{2 \ln(2)} \sigma\\
    E_{\text{FWHM}} & = m \cdot N_{\text{FWHM}}\\
\end{align*}

\begin{figure}[H]
    \centering
    \includegraphics[width=0.8\textwidth]{plots/Cs_peak_gauss.pdf}
    \caption{Fit of the Cs Peak with marked FWHM and FWTM.}
    \label{fig:Cs_peak}
\end{figure}


\begin{figure}[H]
    \centering
    \includegraphics[width=0.8\textwidth]{plots/Cs_compton.pdf}
    \caption{Compton continuum and backscatter peak of the $^{137}$Cs source.}
    \label{fig:Cs_peak_data}
\end{figure}

Theory values:
\begin{align*}
    E_{\text{Compton-Edge}} & = 478,28 \text{keV}\\
    E_{\text{Backscatter}} & = 184,40 \text{keV}
\end{align*}

Experimental values (channel uncertainties: $\approx 10$):
\begin{align*}
    E_{\text{Compton-Edge}} & = 466,18 \pm 0,12 \text{keV}\\
    E_{\text{Backscatter}} & = 192,81 \pm 0,12 \text{keV}
\end{align*}


\begin{figure}
    \centering
    \includegraphics[width=0.8\textwidth]{plots/Cs_linear.pdf}
    \caption{Linear fit of the Compton continuum of the $^{137}$Cs source.}
    \label{fig:Cs_compton_fit}
\end{figure}

\begin{align*}
    m & = (-1,66 \pm 0,33) \cdot 10^{-7} \text{1/s}\\
    b & = (3,84 \pm 0,09) \cdot 10^{-3make} \text{1/s}
\end{align*}

\subsection{Activity Determination}

\begin{figure}[H]
    \centering
    \includegraphics[width=0.8\textwidth]{plots/Ba_spectrum.pdf}
    \caption{Spectrum of the $^{133}$Ba source and the background normed by the time.}
    \label{fig:Ba_spectrum}
\end{figure}

\begin{table}
    \centering
    \caption{Energies wirh greatest probability of Emission for $^{133}$Ba}
    \label{tab:Ba_peaks}
    \begin{tabular}{c c c c}
        \toprule
        Channel & Energy [keV] & $Q$ & Aktivität A [Bq]\\
        \midrule
        688 & 80,99 \pm 0,11 & \\
        2327 & 276,40 \pm 0,12 & \\
        2550 & 302,85 \pm 0,05 & \\
        2997 & 356,01 \pm 0,07 & \\
        3226 & 383,85 \pm 0,12 & \\
        \bottomrule
    \end{tabular}
\end{table}

\subsection{Nuclide Identification}
\begin{figure}[H]
    \centering
    \includegraphics[width=0.8\textwidth]{plots/Unknown_spectrum.pdf}
    \caption{Identified peaks of the $^{133}$Ba source.}
    \label{fig:Ba_peaks}
\end{figure}

