\section{Auswertung}
\label{sec:Auswertung}
Alle Berechnungen werden mit dem Programm \glqq NumPy" \cite{numpy}, die Unsicherheiten mit dem Modul \glqq Uncertainties" \cite{uncertainties}, die Ausgleichsrechnungen mit dem Modul \glqq Scipy" \cite{scipy} durchgeführt und die grafischen Darstellungen über das Modul \glqq Matplotlib" \cite{matplotlib} erstellt.

\subsection{Intensitätsverteilung}

Als erstes wird die maximale Intensität $I_0$ anhand des Detektorscans bestimmt. Hier wird eine Gaußfunktion über die Curvefit Funktion aus Scipy\ref{uncertainties}
gefittet. 

\begin{figure}[H]
    \centering
    \includegraphics[width=0.8\textwidth]{plots/dscan.pdf}
    \caption{Detektorscan}
    \label{fig:dscan}
\end{figure}

Die folgenden Parameter wurden somit bestimmt und die $FWHM = 2\sqrt{2\cdot\ln{2}}\cdot\sigma$ berechnet:
\begin{align}
    I_{\max} &= 228603 \;\text{s}^{-1} \\
    \mu      &= (-0.0014 \pm 0.0005)^{\circ} \\
    \sigma   &= (0.0382 \pm 0.0005)^{\circ} \\
    \text{FWHM} &= 0.0900^{\circ}
\end{align}
    
\subsection{Strahlbreite}

Anhand des Z-Scans kann die Strahlbreite abgeschätzt werden, indem der Bereich, in dem die gemessene Intensität stetig abnimmt, den Messdaten entnommen wird. 
In Abbildung\ref{fig:dscan} ist das entsprechende Intervall zu sehen. Die Strahlbreite ergibt sich somit zu $d = \SI{0.123}{\milli\metre}$

\begin{figure}[H]
    \centering
    \includegraphics[width=0.8\textwidth]{plots/zscan.pdf}
    \caption{Zscan}
    \label{fig:zscan}
\end{figure}

Darüber hinaus wird anhand des X-Scans die Probenbreite abgeschätzt. Dazu wird das Intervall, in dem die Intensität zwischen beiden Plateaus liegt, den Daten entnommen.
Die Probenbreite ergibt sich damit zu $D =\SI{22.489}{\milli\metre}$ und ist in Abbildung\ref{fig:xscan} visualisiert.
\begin{figure}[H]
    \centering
    \includegraphics[width=0.8\textwidth]{plots/xscan.pdf}
    \caption{Xscan}
    \label{fig:xscan}
\end{figure}

Außerdem wird der Geometriewinkel $\alpha_G$ mithilfe des Rockingscans bestimmt. Die Probe wird hierbei im Strahl rotiert, wobei die Intensität gegen den Rotationswinkel
aufgetragen wird. Die beiden Grenzregionen, an denen die Intensität auf null fällt, kennzeichnen den Geometriewinkel. Dieser beträgt hier $\alpha_G = \SI{0.39}{\degree}$,
was in Abbildung~\ref{fig:rscan} visualisiert ist. 
\begin{figure}[H]
    \centering
    \includegraphics[width=0.8\textwidth]{plots/rscan.pdf}
    \caption{Rockingscan}
    \label{fig:rscan}
\end{figure}

$\alpha_G$ lässt sich auch rechnerisch über die bereits bestimmten Strahl- und Probenbreite anhand von Gleichung\ref{eq:breite} bestimmen. 
Damit ergeben sich, je nach gewählter Probenbreite
\begin{align*}
    \alpha_{G_{th}} = \SI{0.3527}{\circ}&\, \text{with} D_{th}= \SI{20.0}{\milli\metre}& \\
    \alpha_{G_{ex}} = \SI{0.3137}{\circ}&\, \text{with} D_{ex}= \SI{22.49}{\milli\metre}& 
\end{align*}

\subsection{Reflektivitätsmessung}

Nach Normierung der Intensität anhand der errechneten maximalen Intensität kann eine Reflektivität in Abhängikeit des Einfallswinkels der Röntgenstrahlung zur Probe 
dargestellt werden. Diese Messung wurde sowohl mit dem Detektor immer im Ausfallswinkel, als auch mit dem Detektor in einer zum Ausfallswinkel verschobenen Position 
durchgeführt. Der somit ermittelte Einfluss diffuser Streuung kann dann von der eigentlichen Messung subtrahiert werden. Dadurch ergibt sich dann die reine 
winkelabhängige Reflektivität. Beide unangepassten Verläufe sind in Abbildung~\ref{fig:ref_raw} dargestellt, sowie die angepasste Reflektivität im Vergleich zur
Fresnel-Kurve für eine optimal glatte Siliziumoberfläche in Abbildung~\ref{fig:ref_mod}.

\begin{figure}[H]
    \centering
    \includegraphics[width=0.8\textwidth]{plots/ref_raw.pdf}
    \caption{Reflektivität und Diffuse Streuung.}
    \label{fig:ref_raw}
\end{figure}

\begin{figure}[H]
    \centering
    \includegraphics[width=0.8\textwidth]{plots/ref_mod.pdf}
    \caption{reduzierte Reflektivität und Fresnel-Kurve.}
    \label{fig:ref_mod}
\end{figure}

Für kleine Winkel, für die nicht die gesamte Breite des Strahls die Probe trifft, wird die Reflektivität durch den Geometriefaktor aus Gleichung~\ref{eq:g_factor} korrigiert.
In Abbildung~\ref{fig:g_corr} ist die korrigierte Reflektivität für Winkel $ \alpha_g < \SI{0.3527}{\degree}$ dargestellt.

\begin{figure}[H]
    \centering
    \includegraphics[width=0.8\textwidth]{plots/g_corr.pdf}
    \caption{Reflektivität mit und ohne Geometriefaktorkorrektur.}
    \label{fig:g_corr}
\end{figure}

Mit der korrigierten Reflektivität werden alle weiteren Auswertungsschritte durchgeführt. Zunächst wird die Schichtdicke des Polystyrol-Films über die Gleichung~\ref{eq:d_pol} 
abgeschätzt. Dabei wurde $ q_z = \frac{4\pi}{\lambda}\sin(\alpha) $ benutzt, um im $q$-Raum die Amplituden der Kiessig-Oszillationen zu finden. Dies wurde mit der 
Scipy Funktion FindPeaks durchgeführt. Die ermittelten Amplituden sind in Abbildung~\ref{fig:peaks} zu sehen. Mit dem so bestimmten mittleren Amplitudenabstand 
$\Delta q_z = 0.0071\pm0.0005$ kann die Schichtdicke zu $d = \SI{88.27\pm6.24}{\angstrom}$ bestimmt werden, was eine erste Abschätzung darstellt. 

\begin{figure}[H]
    \centering
    \includegraphics[width=0.8\textwidth]{plots/kiessig.pdf}
    \caption{Amplituden der Kiessig Oszillationen.}
    \label{fig:peaks}
\end{figure}

Wie im Unterabschnitt~\ref{sec:multi} beschrieben, können mit dem Parratt-Algorithmus weitere Eigenschaften der Probe anhand der Reflektivitätsmessung bestimmt werden.
Dabei werden die Rauigkeiten $\sigma_\text{Sil}$ und $\sigma_\text{Pol}$, die Dispersionskoeffizienten $\delta\text{Sil}$ und $\delta\text{Pol}$, die 
Absorptionskoeffizienten $\beta\text{Sil}$ und $\beta\text{Pol}$, sowie die Schichtdicke $d_\text{Pol}$ bestimmt. Die Fitfunktion ist in Abbildung~\ref{fig:parratt}
dargestellt und die berechneten Parameter in der Tabelle~\ref{tab:parratt}.

\begin{figure}[H]
    \centering
    \includegraphics[width=0.8\textwidth]{plots/parratt.pdf}
    \caption{Reflektivität mit Parratt Fit.}
    \label{fig:parratt}
\end{figure}

\begin{table}[h]
    \centering
    \caption{Parratt-Fit Parameter im Vergleich zu Theoriewerten.}
    \label{tab:parratt_theory}
    \begin{tabular}{l c c}
    \toprule
    \textbf{Parameter} & \textbf{Parratt Algorithmus} & \textbf{Theoriewert} \\
    \midrule
    $\delta_{\mathrm{Si}}$      & $7.971 \cdot 10^{-6}$ & $7.6 \cdot 10^{-6}$ \\
    $\delta_{\mathrm{Pol}}$     & $2.189 \cdot 10^{-6}$ & $3.5 \cdot 10^{-6}$ \\
    $\sigma_{\mathrm{Si}}$      & $15.13\,\AA$          & $10\,\AA$ \\
    $\sigma_{\mathrm{Pol}}$     & $23.31\,\AA$          & $10\,\AA$ \\
    $\beta_{Si}$                 & $2.227 \cdot 10^{-7}$ & $1.9 \cdot 10^{-7}$ \\
    $\beta_{Pol}$                 & $1.019 \cdot 10^{-7}$ & $1.75 \cdot 10^{-8}$ \\
    $d$                         & $87.80\,\mathrm{nm}$  & $8.9 \cdot 10^{-8}\,\mathrm{m}$ \\
    \bottomrule
    \end{tabular}
    \label{tab:parratt}
    \end{table}
    

