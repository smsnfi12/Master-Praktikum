\section{Theorie}
\label{sec:Theorie}
Als Reflektometrie wird die Untersuchung der Reflexion von elektromagnetischen Wellen an Grenzflächen zwischen verschiedenen Medien bezeichnet. 
Röntgenreflektometrie ist ein zerstörungsfreies Analyseverfahren, bei der Röntgenstrahlung verwendet wird.
Bei Röntgenstrahlung handelt es sich um elektromagnetische Strahlung mit Energien oberhalb von 100 eV, was einer Wellenlänge von unter 10 nm entspricht.

\subsection{Erzeugung von Röntgenstrahlung}
Röntgenstrahlung wird in einer Röntgenröhre erzeugt, die aus einer Glühkathode und einer Kupferanode besteht.
Der Aufbau einer Röntgenröhre ist in Abbildung \ref{fig:roentgenroehre} dargestellt.\\
\begin{figure}
    \centering
    \includegraphics[width=0.4\textwidth]{Abbildungen/Röntgenröhre.png}
    \caption{Schematischer Aufbau einer Röntgenröhre.}
    \label{fig:roentgenroehre}
\end{figure}

Durch thermische Emission werden Elektronen von der Kathode freigesetzt und durch eine angelegte Spannung $U_\text{B}$ auf die Anode beschleunigt.
Treffen die Elektronen auf die Anode, so werden sie abgebremst und geben dabei Energie in Form von Röntgenstrahlung ab.
Es wird zwischen Bremsstrahlung und charakteristischer Röntgenstrahlung unterschieden:
\begin{itemize}
    \item \textbf{Bremsstrahlung:} Die Bremsstrahlung entsteht durch das Abbremsen der Elektronen im Coulombfeld der Atomkerne der Anode.
     Da die Elektronen unterschiedlich stark abgebremst werden, werden dabei Röntgenphotonen mit verschiedenen Energien emittiert.
     Dies führt zu einem kontinuierlichen Spektrum der Bremsstrahlung.
     Die maximale Energie der Bremsstrahlung entspricht der angelegten Spannung zwischen Kathode und Anode $U_\text{B}$.
    \item \textbf{Charakteristische Röntgenstrahlung:} Die beschleunigten Elektronen können auch Elektronen aus den inneren Schalen der Atome der Anode herausschlagen.
    Dabei entstehen Löcher, die Elektronen aus höheren Schalen füllen.
    Dieser Übergang führt zur Emission von Röntgenstrahlung mit diskreten Energien, die charakteristisch für das Anodenmaterial sind.
    Die Energien der charakteristischen Linien entsprechen den Energiedifferenzen der Elektronenschalen.
\end{itemize}
\begin{figure}
    \centering
    \includegraphics[width=0.5\textwidth]{Abbildungen/Spektrum.png}
    \caption{Typisches Röntgenspektrum mit Bremsstrahlung und charakteristischen Linien.}
    \label{fig:spektrum}
\end{figure}
In Abbildung \ref{fig:spektrum} ist ein typisches Röntgenspektrum dargestellt, das sowohl die kontinuierliche Bremsstrahlung als auch die diskreten charakteristischen Linien zeigt.
Die Linien $K_\alpha$ und $K_\beta$ entstehen bei den Übergängen von der L- bzw. M-Schale zur K-Schale.


\subsection{Röntgenstrahlung an Grenzflächen}
Fällt Strahlung auf eine Grenzfläche zwischen zwei Medien, so wird ein Teil der Strahlung reflektiert und ein Teil transmittiert (siehe Abbildung \ref{fig:grenzflaeche}).
Im Falle von Röntgenstrahlung gilt für den Brechungsindex eines Mediums
\begin{equation*}
    n = 1 - \delta +i\beta,
\end{equation*}
wobei $\delta$ die Dispersion und $\beta$ die Absorption des Mediums beschreiben.
\begin{figure}
    \centering
    \includegraphics[width=0.6\textwidth]{Abbildungen/Fresnel.png}
    \caption{Reflexion und Transmission an einer Grenzfläche zwischen zwei Medien.}
    \label{fig:grenzflaeche}
\end{figure}
Der Dispersionsanteil $\delta$ entsteht durch die Wechselwirkung der Röntgenstrahlung mit der Elektronen des Materials und beeinflusst die Phasengeschwindigkeit der Welle.
Die einfallende elektromagnetische Welle beschleunigt die Elektronen durch elastische Streuung, wodurch diese als Dipole agieren und selbst elektromagnetische Wellen abstrahlen.
Die Überlagerung der primären Welle mit den von den Elektronen emittierten Wellen führt zu einer Phasenverschiebung, die größer als die Lichtgeschwindigkeit im Vakuum sein kann.
Die Dispersion $\delta$ ist proportional zur Elektronendichte des Materials $\rho_e$ und ist gegeben durch
\begin{equation*}
    \delta = \frac{r_e}{2 \pi} \lambda^2 \rho_e
\end{equation*}
mit dem klassischen Elektronenradius $r_e$ und der Wellenlänge $\lambda$ der Röntgenstrahlung.
Die Dispersion $\delta$ ist für Röntgenstrahlung sehr klein (in Größenordnung von $10^{-6}$), weshalb der Brechungsindex $n$ für Röntgenstrahlung in der Regel kleiner als 1 ist.\\
Der Absorptionsanteil $\beta$ beschreibt die Energieverluste im Material durch inelastische Wechselwirkungen und ist gegeben durch
\begin{equation*}
    \beta = \frac{\mu \lambda}{4 \pi},
\end{equation*}
wobei $\mu$ der Absorptionskoeffizient des Materials ist.\\


Beim Übergang vom Vakuum in ein Medium mit Brechungsindex $n$ werden folgende Randbedingungen berücksichtigt:
\begin{align*}
    a_I + a_R &= a_T, \\
    \vec{k}_I a_I - \vec{k}_R a_R &= \vec{k}_T a_T.
\end{align*}
Hierbei sind $a_I$, $a_R$ und $a_T$ die Amplituden der einfallenden, reflektierten und transmittierten Welle, während $k_I$, $k_R$ und $k_T$ die entsprechenden Wellenvektoren sind.
Mit dem Wellenvektor im Vakuum $k = |\vec{k}_I| =|\vec{k}_R|$ und dem Wellenvektor im Medium $|\vec{k}_T| = n k$ kann die zweite Randbedingung in parallelen und senkrechten Komponenten  bezogen auf die Oberfläche zerlegt werden:
\begin{align*}
    (a_I + a_R) k \cos\alpha &= a_T (n k) \cos\alpha ', \\
   (a_I  - a_R) k \sin\alpha &= a_T (n k) \sin\alpha '.
\end{align*}
Daraus lässt sich das Snellius'sche Brechungsgesetz
\begin{equation}
    \cos \alpha = n \cos \alpha '
    \label{eq:snellius}
\end{equation}
ableiten.
Für kleine Winkel $\alpha$ und $\alpha '$ folgt
\begin{equation*}
    \frac{a_I-a_R}{a_I+a_R} = n \frac{\sin\alpha '}{\sin\alpha} \approx \frac{\alpha '}{\alpha}.
\end{equation*}
Daraus ergeben sich die Fresnel'schen Gleichungen für die Reflexions- und Transmissionskoeffizienten $r$ und $t$:
\begin{align*}
    r &= \frac{a_R}{a_I} = \frac{\alpha - \alpha '}{\alpha + \alpha '}, \\
    t &= \frac{a_T}{a_I} = \frac{2 \alpha}{\alpha + \alpha '}.
\end{align*}
Die Intensität der reflektierten und transmittierten Welle ist durch das Betragsquadrat der Amplituden $r$ und $t$ gegeben.\\
Totalreflexion tritt auf, wenn der Einfallswinkel $\alpha$ kleiner als der kritische Winkel $\alpha_c$ ist.
Zur Bestimmung des kritischen Winkels wird die Bedingung für die Totalreflexion ($\alpha '=0$) in die Snellius'sche Gleichung \eqref{eq:snellius} eingesetzt, was zu
\begin{equation*}
    \cos \alpha_c = n = 1 - \delta
\end{equation*}
führt, wobei der Absorptionsanteil $\beta$ vernachlässigt wird.
Für kleine Winkel kann der kritische Winkel zu
\begin{equation*}
    \alpha_c \approx \sqrt{2 \delta} 
\end{equation*}
genähert werden.
Für Einfallswinkel $\alpha$ deutlich größer als den kritischen Winkel $\alpha_c$ ergbit sich für den Reflexionskoeffizienten die Näherung
\begin{equation*}
    R = |r|^2 \approx \left( \frac{\alpha_c}{2 \alpha} \right)^4.
\end{equation*}
Daraus lässt sich erkennen, dass die Intensität der reflektierten Welle mit steigendem Einfallswinkel stark abnimmt.

\subsection{Multischichtsysteme}









\subsection{Geometriefaktor}