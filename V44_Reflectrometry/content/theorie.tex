\section{Theorie}
\label{sec:Theorie}
Als Reflektometrie wird die Untersuchung der Reflexion von elektromagnetischen Wellen an Grenzflächen zwischen verschiedenen Medien bezeichnet. 
Röntgenreflektometrie ist ein zerstörungsfreies Analyseverfahren, bei dem Röntgenstrahlung verwendet wird.
Bei Röntgenstrahlung handelt es sich um elektromagnetische Strahlung mit Energien oberhalb von 100 eV, was einer Wellenlänge von unter 10 nm entspricht.

\subsection{Erzeugung von Röntgenstrahlung}
Röntgenstrahlung wird in einer Röntgenröhre erzeugt, die aus einer Glühkathode und einer Kupferanode besteht.
Der Aufbau einer Röntgenröhre ist in Abbildung \ref{fig:roentgenroehre} dargestellt.\\
\begin{figure}
    \centering
    \includegraphics[width=0.4\textwidth]{Abbildungen/Röntgenröhre.png}
    \caption{Schematischer Aufbau einer Röntgenröhre \cite{xray}.}
    \label{fig:roentgenroehre}
\end{figure}

Durch thermische Emission werden Elektronen von der Kathode freigesetzt und durch eine angelegte Spannung $U_\text{B}$ auf die Anode beschleunigt.
Treffen die Elektronen auf die Anode, so werden sie abgebremst und geben dabei Energie in Form von Röntgenstrahlung ab.
Es wird zwischen Bremsstrahlung und charakteristischer Röntgenstrahlung unterschieden:
\begin{itemize}
    \item \textbf{Bremsstrahlung:} Die Bremsstrahlung entsteht durch das Abbremsen der Elektronen im Coulombfeld der Atomkerne der Anode.
     Da die Elektronen unterschiedlich stark abgebremst werden, werden dabei Röntgenphotonen mit verschiedenen Energien emittiert.
     Dies führt zu einem kontinuierlichen Spektrum der Bremsstrahlung.
     Die maximale Energie der Bremsstrahlung entspricht der angelegten Spannung zwischen Kathode und Anode $U_\text{B}$.
    \item \textbf{Charakteristische Röntgenstrahlung:} Die beschleunigten Elektronen können auch Elektronen aus den inneren Schalen der Atome der Anode herausschlagen.
    Dabei entstehen Löcher, die Elektronen aus höheren Schalen füllen.
    Dieser Übergang führt zur Emission von Röntgenstrahlung mit diskreten Energien, die charakteristisch für das Anodenmaterial sind.
    Die Energien der charakteristischen Linien entsprechen den Energiedifferenzen der Elektronenschalen.
\end{itemize}
\begin{figure}
    \centering
    \includegraphics[width=0.5\textwidth]{Abbildungen/Spektrum.png}
    \caption{Typisches Röntgenspektrum mit Bremsstrahlung und charakteristischen Linien \cite{xray}.}
    \label{fig:spektrum}
\end{figure}
In Abbildung \ref{fig:spektrum} ist ein typisches Röntgenspektrum dargestellt, das sowohl die kontinuierliche Bremsstrahlung als auch die diskreten charakteristischen Linien zeigt.
Die Linien $K_\alpha$ und $K_\beta$ entstehen bei den Übergängen von der L- bzw. M-Schale zur K-Schale.


\subsection{Röntgenstrahlung an Grenzflächen}
Fällt Strahlung auf eine Grenzfläche zwischen zwei Medien, so wird ein Teil der Strahlung reflektiert und ein Teil transmittiert (siehe Abbildung \ref{fig:grenzflaeche}).
Im Falle von Röntgenstrahlung gilt für den Brechungsindex eines Mediums
\begin{equation*}
    n = 1 - \delta +i\beta,
\end{equation*}
wobei $\delta$ die Dispersion und $\beta$ die Absorption des Mediums beschreiben.
\begin{figure}
    \centering
    \includegraphics[width=0.6\textwidth]{Abbildungen/Fresnel.png}
    \caption{Reflexion und Transmission an einer Grenzfläche zwischen zwei Medien \cite{xray}.}
    \label{fig:grenzflaeche}
\end{figure}
Der Dispersionsanteil $\delta$ entsteht durch die Wechselwirkung der Röntgenstrahlung mit den Elektronen des Materials und beeinflusst die Phasengeschwindigkeit der Welle.
Die einfallende elektromagnetische Welle beschleunigt die Elektronen durch elastische Streuung, wodurch diese als Dipole agieren und selbst elektromagnetische Wellen abstrahlen.
Die Überlagerung der primären Welle mit den von den Elektronen emittierten Wellen führt zu einer Phasenverschiebung, die größer als die Lichtgeschwindigkeit im Vakuum sein kann.
Die Dispersion $\delta$ ist proportional zur Elektronendichte des Materials $\rho_e$ und ist gegeben durch
\begin{equation*}
    \delta = \frac{r_e}{2 \pi} \lambda^2 \rho_e
\end{equation*}
mit dem klassischen Elektronenradius $r_e$ und der Wellenlänge $\lambda$ der Röntgenstrahlung.
Die Dispersion $\delta$ ist für Röntgenstrahlung sehr klein (in Größenordnung von $10^{-6}$), weshalb der Brechungsindex $n$ für Röntgenstrahlung in der Regel kleiner als 1 ist.\\
Der Absorptionsanteil $\beta$ beschreibt die Energieverluste im Material durch inelastische Wechselwirkungen und ist gegeben durch
\begin{equation*}
    \beta = \frac{\mu \lambda}{4 \pi},
\end{equation*}
wobei $\mu$ der Absorptionskoeffizient des Materials ist.\\


Beim Übergang vom Vakuum in ein Medium mit Brechungsindex $n$ werden folgende Randbedingungen berücksichtigt:
\begin{align*}
    a_I + a_R &= a_T, \\
    \vec{k}_I a_I - \vec{k}_R a_R &= \vec{k}_T a_T.
\end{align*}
Hierbei sind $a_I$, $a_R$ und $a_T$ die Amplituden der einfallenden, reflektierten und transmittierten Welle, während $k_I$, $k_R$ und $k_T$ die entsprechenden Wellenvektoren sind.
Mit dem Wellenvektor im Vakuum $k = |\vec{k}_I| =|\vec{k}_R|$ und dem Wellenvektor im Medium $|\vec{k}_T| = n k$ kann die zweite Randbedingung in parallelen und senkrechten Komponenten  bezogen auf die Oberfläche zerlegt werden:
\begin{align*}
    (a_I + a_R) k \cos\alpha &= a_T (n k) \cos\alpha ', \\
   (a_I  - a_R) k \sin\alpha &= a_T (n k) \sin\alpha '.
\end{align*}
Daraus lässt sich das Snelliussche Brechungsgesetz
\begin{equation}
    \cos \alpha = n \cos \alpha '
    \label{eq:snellius}
\end{equation}
ableiten.
Für kleine Winkel $\alpha$ und $\alpha '$ folgt
\begin{equation*}
    \frac{a_I-a_R}{a_I+a_R} = n \frac{\sin\alpha '}{\sin\alpha} \approx \frac{\alpha '}{\alpha}.
\end{equation*}
Daraus ergeben sich die Fresnelschen Gleichungen für die Reflexions- und Transmissionskoeffizienten $r$ und $t$:
\begin{align*}
    r &= \frac{a_R}{a_I} = \frac{\alpha - \alpha '}{\alpha + \alpha '}, \\
    t &= \frac{a_T}{a_I} = \frac{2 \alpha}{\alpha + \alpha '}.
\end{align*}
Die Intensität der reflektierten und transmittierten Welle ist durch das Betragsquadrat der Amplituden $r$ und $t$ gegeben.\\
Totalreflexion tritt auf, wenn der Einfallswinkel $\alpha$ kleiner als der kritische Winkel $\alpha_c$ ist.
Zur Bestimmung des kritischen Winkels wird die Bedingung für die Totalreflexion ($\alpha '=0$) in die Snelliussche Gleichung \eqref{eq:snellius} eingesetzt, was zu
\begin{equation*}
    \cos \alpha_c = n = 1 - \delta
\end{equation*}
führt, wobei der Absorptionsanteil $\beta$ vernachlässigt wird.
Für kleine Winkel kann der kritische Winkel zu
\begin{equation*}
    \alpha_c \approx \sqrt{2 \delta} 
\end{equation*}
genähert werden.
Für Einfallswinkel $\alpha$ deutlich größer als den kritischen Winkel $\alpha_c$ ergbit sich für den Reflexionskoeffizienten die Näherung
\begin{equation*}
    R = |r|^2 \approx \left( \frac{\alpha_c}{2 \alpha} \right)^4.
\end{equation*}
Daraus lässt sich erkennen, dass die Intensität der reflektierten Welle mit steigendem Einfallswinkel stark abnimmt.

\subsection{Multischichtsysteme}
\label{sec:multi}
Multischichtsysteme bestehen aus mehreren dünnen Schichten unterschiedlicher Materialien, die auf einem Substrat aufgebracht sind.\\
An jeder Grenzfläche zwischen zwei Schichten kommt es zu Reflexion und Transmission der Röntgenstrahlung.
Die Gesamtreflexion und -transmission eines Multischichtsystems ergibt sich aus der Überlagerung der an den einzelnen Grenzflächen reflektierten und transmittierten Wellen.
Der schematische Aufbau eines Multischichtsystems ist in Abbildung \ref{fig:multischicht} dargestellt.\\
\begin{figure}[H]
    \centering
    \includegraphics[width=0.6\textwidth]{Abbildungen/Multischichtsysteme.png}
    \caption{Reflexion und Transmission an den Grenzflächen eines Multischichtsystems \cite{xray}.}
    \label{fig:multischicht}
\end{figure}
Bei der Überlagerung der an der Ober- und Unterseite der Schicht reflektierten Wellen treten Interferenzeffekte auf.
Diese führen zu periodischen Oszillationen im Reflexionsspektrum, die als Kiessig-Oszillationen bezeichnet werden.
In Abbildung \ref{fig:kiessig} sind typische Kiessig-Oszillationen dargestellt.
\begin{figure}
    \centering
    \includegraphics[width=0.6\textwidth]{Abbildungen/Kiessig.png}
    \caption{Typische Kiessig-Oszillationen im Reflexionsspektrum eines Multischichtsystems \cite{xray}.}
    \label{fig:kiessig}
\end{figure}
Die Periode der Kiessig-Oszillationen ist gegeben durch die Dicke der Schicht $d$
\begin{equation*}
    d = \frac{\lambda}{2 \Delta \alpha _I},
    \label{eq:breite}
\end{equation*}
die von der Differenz der Einfallswinkel $\Delta \alpha$ der aufeinanderfolgenden Maxima und der Wellenlänge abhängt.\\


Zur Berechnung der Reflexionskoeffizienten von Multischichtsystemen wird der Parratt-Algorithmus verwendet.
Der Parratt-Algorithmus ist ein rekursives Verfahren, das die Schichtdicken, Materialdichten und Rauigkeiten berücksichtigt.
Das Multischichtsystem besteht aus $N$ Schichten, wobei jede Schicht durch ihren Brechungsindex $n_j$ und ihre Dicke $d_j$ beschrieben wird.
Das Verfahren beginnt mit der Berechnung der Reflexions- und Transmissionskoeffizienten an der untersten Grenzfläche, die direkt auf dem Substrat liegt.
Dabei wird angenommen, dass das Substrat unendlich dick ist, sodass der Reflexionskoeffizient an der Unterseite verschwindet.
Das Verhältnis der reflektierten Amplitude $R_j$ und der transmittierten Amplitude $T_j$ ist gegeben durch
\begin{equation*}
    X_j = \frac{R_j}{T_j} = \exp(-2i k_{z,j} d_j) \frac{r_{j,j+1}+X_{j+1}\exp(2ik_{z,j+1}z_j)}{1+r_{j,j+1}X_{j+1}\exp(2ik_{z,j+1}z_j)},
\end{equation*}
wobei $z_j$ die Position der j-ten Schicht und $r_{j,j+1}$ der Fresnelkoeffizient der j-ten Grenzfläche beschreiben.
Die z-Komponente des Wellenvektors in der j-ten Schicht ist gegeben durch
\begin{equation*}
    k_{z,j} = k \sqrt{n_j^2 - \cos^2(\alpha)}
\end{equation*}
mit dem Einfallswinkel $\alpha$. Für die Rauigkeit $\sigma_j$ gilt
\begin{equation*}
    \sigma^2_j = \int (z-z_j)^2 P_j(z) dz,
\end{equation*}
wobei $P_j(z)$ die Wahrscheinlichkeitsverteilung der Rauigkeit an der j-ten Grenzfläche beschreibt.
Daraus lassen sich die Fresnelkoeffizienten ableiten:
\begin{align*}
    r_{j,j+1} &= r_{j,j+1} \exp(-2 k_{z,j} k_{z,j+1} \sigma^2_j), \\
    t_{j,j+1} &= t_{j,j+1} \exp\left(\frac{(k_{z,j} - k_{z,j+1})^2 \sigma^2_j}{2}\right).
\end{align*}
Diese werden in die rekursive Gleichung für $X_j$ eingesetzt, um die Reflexions- und Transmissionskoeffizienten für das gesamte Multischichtsystem zu berechnen.

\subsection{Geometriefaktor}
Wird der Einfallswinkel sehr klein gewählt, so kann es passieren, dass der Röntgenstrahl die Probenoberfläche nicht vollständig trifft.
In diesem Fall wird ein Geometriefaktor $G$ eingeführt, der das Verhältnis der tatsächlich beleuchteten Probenfläche zur gesamten Probenfläche beschreibt.
Der Geometriefaktor ist gegeben durch
\begin{equation*}
    G(\alpha) =
\begin{cases}
\dfrac{D \sin(\alpha)}{d} & \text{für } \alpha < \alpha_g , \\
1                         & \text{für } \alpha \ge \alpha_g ,
\end{cases}
\label{eq:g_factor}
\end{equation*}
wobei $D$ die Länge der Probe und $d$ die Strahlbreite ist.
$\alpha_g$ bezeichnet den geometrischen Einfallswinkel, bei dem die gesamte Probenoberfläche beleuchtet wird.
Er ist gegeben durch
\begin{equation*}
    \alpha_g = \arcsin\left(\frac{d}{D}\right).
\end{equation*}
