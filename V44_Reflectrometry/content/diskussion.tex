\section{Diskussion}
\label{sec:Diskussion}

Die Kalibirierungsmessungen entsprechen im Allgemeinen den Erwartungen. Die über den X-scan ermittelte Probenbreite von $D =\SI{22.489}{\milli\metre}$ weicht um 
$\SI{12.45}{\percent}$ vom Theoriewert $D = \SI{20.0}{\milli\metre}$ ab. Das könnte darauf zurückzuführen sein, dass die Grenzen durch die zu grobe Messrate nicht präzise 
genug bestimmt werden konnten. \newline\newline
Die durch den Parratt-Algorithmus ermittelten Daten entsprechen nur Teilweise den Theoriewerten zu einer akzeptablen Genauigkeit. Die Diffusionskoeffizienten, sowie 
der Absorptionskoeffizient der Siliziumoberfläche, liegen nah am Theoriewert. Außerdem stimmt die ermittelte Schichtdicke des Films mit einer Abweichung von $\SI{1,35}{\percent}$
mit der zuvor abgeschätzten Schichtdicke überein.\newline
Signifikante Abweichungen zeigen allerdings die Rauigkeiten $\sigma$ an beiden Grenzflächen, sowie die Absorption an der Luft-Film Oberfläche. 
Diese Abweichungen der Parameter sind auf einen nicht optimalen Fit zurückzuführen, was wiederum auf ein signikantes Problem bei der Reflektivitätsmessung 
zurückzuführen ist. Bei dieser ist deutlich sichtbar, dass zwischen $0,6\circ$ und $0,8\circ$ eine zusätzliche, Signifikante Reflexion zustande kommt, welche nicht 
auf Kiessig Oszillationen zurückzuführen ist, da diese ebenso bei der Messung der diffusen Streuung auftritt. Die naheliegende Erklärung hierfür ist eine Reflexion 
an einer Oberfläche eines der in diesem Winkelbereich im Weg des Strahls liegenden Intstrumente, die zur Halterung der Probe gehören könnten.\newline
Dies hat zufolge, dass der Parratt-Fit nur bis zum Winkel dieses Ereignisses durchgeführt werden konnte. Das Ausschneiden des Intervalls für den Fit gab keine 
physikalisch realistischen Parameter.  

