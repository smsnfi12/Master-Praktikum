\section{Analysis}
\label{sec:Auswertung}

All calculations are performed using the NumPy \cite{numpy} library, the uncertainties are handled with the Uncertainties \cite{uncertainties} module, 
the curve fitting is carried out with the SciPy \cite{scipy} module, and the graphical representations are created using the Matplotlib \cite{matplotlib} 
library.

\subsection{Contrast Measurement}

\begin{table}[h!]
    \centering
    \caption{Measured maximum and minimum diode voltages $I_{\text{max}}$ and $I_{\text{min}}$ for different polarizer angles.}
    \label{tab:contrast_measurement}
    \begin{tabular}{c|cc|cc|cc}
    \toprule
    \textbf{Angle} & \multicolumn{2}{c|}{\textbf{Run 1}} & \multicolumn{2}{c|}{\textbf{Run 2}} & \multicolumn{2}{c}{\textbf{Run 3}} \\
    $^{\circ}$ & $I_{\text{max,1}}$/V & $I_{\text{min,1}}$/V & $I_{\text{max,2}}$/V & $I_{\text{min,2}}$/V & $I_{\text{max,3}}$/V & $I_{\text{min,3}}$/V \\
    \midrule
    0   & 1.39 & 1.16 & 1.44 & 1.25 & 1.55 & 1.35 \\
    15  & 1.15 & 0.59 & 1.17 & 0.59 & 1.15 & 0.63 \\
    30  & 0.99 & 0.23 & 1.06 & 0.22 & 1.07 & 0.23 \\
    45  & 1.10 & 0.09 & 1.14 & 0.09 & 1.17 & 0.10 \\
    60  & 1.42 & 0.13 & 1.52 & 0.13 & 1.47 & 0.19 \\
    75  & 1.48 & 0.40 & 1.55 & 0.39 & 1.55 & 0.42 \\
    90  & 1.66 & 1.30 & 1.82 & 1.23 & 1.79 & 1.20 \\
    105 & 2.26 & 1.07 & 2.48 & 1.06 & 2.66 & 1.19 \\
    120 & 3.46 & 0.53 & 3.59 & 0.53 & 3.74 & 0.54 \\
    135 & 4.00 & 0.30 & 4.19 & 0.32 & 4.30 & 0.30 \\
    150 & 3.80 & 0.59 & 3.77 & 0.63 & 3.91 & 0.56 \\
    165 & 2.54 & 1.06 & 2.73 & 1.12 & 2.71 & 0.98 \\
    180 & 1.45 & 1.25 & 1.53 & 1.24 & 1.58 & 1.33 \\
    \bottomrule
    \end{tabular}
\end{table}
    
Depicted in table~\ref{tab:contrast_measurement} are the measured light intensities at different polarizer angles for 3 different runs via a single diode. From these intensities,
the mean values at each angle, the contrasts $V_\text{exp.}(\phi_\text{i})$ were calculated via equation \eqref{eq:contrast}. The results are visualized in figure~\ref{fig:contrast_plot}
alongside an idealized theory distribution of $V_\text{theo.}(\phi)$ from equation \eqref{eq:kontrast_theorie}. 
\begin{figure}[h!]
    \centering
    \includegraphics[width=0.8\textwidth]{python/contrast.pdf}
    \caption{Comparison of measured diode intensity vs. optimal theory curve in relation to polarizer angle.}
    \label{fig:contrast_plot}
\end{figure}

\subsection{Refractive Index of Glass}

For the determination of the refractive index of glass, a pair of rotating glass plates, described in section \ref{sec:theorie_glas}, with thickness $d= \SI{1}{\milli\meter}$,
is placed in one of the laser beams. Rotating the glass plates affects the optical path length of this beam, creating a phase shift in relation to the rotation
angle $\phi$. Rotating the plates from 30° to 40° or vice versa thus leads to a fixed number of interference fringes along the 10° interval, which were counted 
and documented in table\ref{tab:glass_counts}.
\newline A direct connection between the refractive index $n_\text{glass}$ and the mean count of interference fringes $M$ is layed out in equation \eqref{eq:glas_index_n}. The 
measurement thus leads to a refractive index of 
\begin{equation*}
    n_\text{glass} = 1.529\pm0.034
\end{equation*}

\begin{table}[h!]
    \centering
    \caption{Measured interference maxima $M$ for both rotation directions of the glass plates.}
    \label{tab:glass_counts}
    \begin{tabular}{c|cc}
    \toprule
    \textbf{Run No.} & \textbf{Counts $M$ 30° $\rightarrow$ 40°} & \textbf{Counts $M$ 40° $\rightarrow$ 30°} \\
    \midrule
    1 & 32 & 35 \\
    2 & 34 & 32 \\
    3 & 34 & 32 \\
    4 & 33 & 36 \\
    5 & 33 & 32 \\
    \bottomrule
    \end{tabular}
    \end{table}
    

\subsection{Refractive Index of Air}
\begin{table}[h!]
    \centering
    \caption{Measured number of interference maxima $M$ for different pressures in the gas cell.}
    \label{tab:gas_counts}
    \begin{tabular}{c|cccc}
    \toprule
    \textbf{Pressure / mbar} & \textbf{Run 1} & \textbf{Run 2} & \textbf{Run 3} & \textbf{Run 4} \\
    \midrule
    0   & 0  & 0  & 0  & 0  \\
    50  & 2  & 2  & 3  & 2  \\
    100 & 4  & 5  & 5  & 5  \\
    150 & 7  & 7  & 7  & 7  \\
    200 & 9  & 9  & 9  & 9  \\
    250 & 11 & 11 & 11 & 11 \\
    300 & 13 & 13 & 13 & 13 \\
    350 & 15 & 15 & 15 & 15 \\
    400 & 17 & 18 & 18 & 17 \\
    450 & 19 & 20 & 20 & 20 \\
    500 & 22 & 22 & 22 & 22 \\
    550 & 24 & 24 & 24 & 24 \\
    600 & 26 & 26 & 25 & 26 \\
    650 & 28 & 28 & 28 & 28 \\
    700 & 30 & 30 & 30 & 31 \\
    750 & 32 & 32 & 32 & 32 \\
    800 & 34 & 34 & 34 & 34 \\
    850 & 37 & 36 & 37 & 37 \\
    900 & 38 & 38 & 38 & 38 \\
    950 & 40 & 41 & 40 & 40 \\
    981 & 41 & 41 & 41 & 41 \\
    \bottomrule
    \end{tabular}
    \end{table}
    
As depicted in table~\ref{tab:gas_counts}, 4 runs of interference fringe counts for a series of air pressure values were recorded. Via equation \eqref{eq:gas_index}, the count 
number $M$ can be translated into a difference in refractive index $\Delta n$, where $L = 100.0\pm0.1\si{\centi\meter}$ is the length of the gas cell. 
\newline In figure~\ref{fig:refrac_fit}, the calculated refractive indexes are visualized as a function of air pressure. Also, a linear regression of the 
form $n = m\cdot p + b $ was applied to the measured data points, yielding the following parameters:
\begin{align}
    m &= (2.661 \pm 0.015)\times 10^{-7} \\
    b &= (3.226 \pm 0.089)\times 10^{-6}
\end{align}
With these parameters, the refractive index of air at an air pressure of $\SI{981}{\milli\bar}$ at the day of the experiment can be determined to be 
\begin{equation}
    n_\text{air}= 1.000264\pm0.000001
\end{equation}
According to equation \eqref{eq:refractive_index_gas}, the gradient $m$ is given by 
\begin{equation}
    m= \frac{3A}{2RT}\,,
\end{equation}

where $A$ is the molar refraction number, $R= \SI{8.314}{\joule\per\mole\per\kelvin} $ denotes the universal gas constant, and the room temperatur is measured to $T=\SI{294.85}{\kelvin}$.
the molar refraction number can therefore be determined to be 
\begin{equation}
    A = (4.348 \pm 0.025) \times 10^{-6}\,\mathrm{\frac{m^{3}}{mol}}\,.
\end{equation}

\begin{figure}[h!]
    \centering
    \includegraphics[width=0.8\textwidth]{python/refractive_index_vs_pressure.pdf}
    \caption{Refractive index difference $(n-1)$ of air as a function of pressure with linear regression.}
    \label{fig:refrac_fit}
\end{figure}