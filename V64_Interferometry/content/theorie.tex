\section{Theory}
\label{sec:Theorie} 

\subsection{Coherence}
Coherence refers to the correlation between the phases of two waves. Two waves are coherent if they have a constant phase relationship, same frequency, and polarization.
Important is the concept of temporal and spatial coherence. Temporal coherence refers to the time interval over which wave's phase remains predictable.
Spatial coherence, on the other hand, refers to the correlation between the phases of a wave at different points in space.
The degree of coherence $\gamma$ describes how well two waves $\vec{E}_1$ and $\vec{E}_2$ can interfere with each other. It is defined as
\begin{equation}
    \gamma = \frac{\langle \vec{E}_1 \cdot \vec{E}_2 \rangle}{\sqrt{\langle |\vec{E}_1|^2 \rangle \langle |\vec{E}_2|^2 \rangle}}.
\end{equation}
A degree of coherence of $\gamma = 1$ indicates perfect coherence, while $\gamma = 0$ indicates no coherence \cite{Hecht2002Optics}.\\
Polarization of the electromagnetic wave describes the geometrical orientation of the wave's electric field vector. The plane of polarization is perpendicular to the direction of motion of the wave.
Light waves can be linearly, circularly, or elliptically polarized.
Linear polarization describes oscillations of the electric field within a plane, while circular polarization corresponds to a rotation around the direction of propagation.
In elliptical polarization, the tip of the electric field vector describes an ellipse.
A distinction is made between s- and p-polarized light. S-polarized light has its electric field perpendicular to the optical plane, and p-polarized light has its electric field parallel to it.
Two light beams must have the same polarization state to interfere.
\subsection{Contrast}
The contrast $V$ describes the visibility of interference fringes and is defined as
\begin{equation}
    V = \frac{I_\text{max} - I_\text{min}}{I_\text{max} + I_\text{min}},
    \label{eq:contrast}
\end{equation}
where $I_\text{max}$ and $I_\text{min}$ are the maximum and minimum intensities of the interference pattern.
The contrast can take values between 0 and 1, where $V = 1$ indicates perfect visibility of the fringes, and $V = 0$ no visibility.\\
In order to determine the contrast $V$ as a function of the polarization angle $\phi$, the electric fields of the two beams are considered:
\begin{align}
    E_1(t) &= E_0 \cos(\omega t) \cos(\phi), \\
    E_2(t) &= E_0 \cos(\omega t + \delta) \sin(\phi),
\end{align}
where $\delta$ is the phase difference between the two beams and $\phi$ the polarization angle.
The superposition of the two electric fields is given by
\begin{equation}
    E(t) = E_1(t)+ E_2(t) = E_0 \left[ \cos(\omega t) \cos(\phi) + \cos(\omega t + \delta) \sin(\phi) \right].
    \label{eq:elektrisches_feld}
\end{equation}
The intensity $I$ is proportional to the time-averaged square of the electric field:
\begin{align}
    I &\propto \langle E^2(t) \rangle \nonumber \\
    &= \frac{1}{T}\int_0^{T} E^2(t)  dt = \frac{1}{2\pi}\int_0^{2\pi} E^2(u) d(u).
\end{align}
Using the superposition from equation \eqref{eq:elektrisches_feld} and the relations $\langle \cos^2(\omega t) \rangle = \langle \cos^2(\omega t + \delta) \rangle = \frac{1}{2}$ and $\langle \cos(\omega t) \cos(\omega t + \delta) \rangle = \frac{1}{2} \cos(\delta)$, the intensity can be calculated as
\begin{align}
    I & \propto \langle  (E_0 \cos(\omega t) \cos(\phi) + E_0\cos(\omega t + \delta) \sin(\phi) )^2\rangle \nonumber \\
      & = E_0^2 \left[ \cos^2(\phi) \langle \cos^2(\omega t) \rangle + \sin^2(\phi) \langle \cos^2(\omega t + \delta) \rangle + 2\cos(\phi)\sin(\phi) \langle \cos(\omega t) \cos(\omega t + \delta) \rangle \right] \nonumber \\
      & = E_0^2 \left[ \frac{1}{2} \cos^2(\phi) + \frac{1}{2} \sin^2(\phi) + \cos(\phi)\sin(\phi) \cos(\delta) \right] \nonumber \\
      & = \frac{E_0^2}{2} \left[ 1 + \sin(2\phi) \cos(\delta) \right].
\end{align}
The maximum intensity is obtained for $\delta = 2n\pi$ and the minimum intensity for $\delta =(2n+1) \pi$:
\begin{align}
    I_\text{max} &\propto \frac{E_0^2}{2} \left[ 1 + |\sin(2\phi)| \right], \\
    I_\text{min} &\propto \frac{E_0^2}{2} \left[ 1 - |\sin(2\phi)| \right].
\end{align}
Using the equation \eqref{eq:contrast} the contrast $V$ can be expressed as a function of the polarization angle $\phi$
\begin{equation}
    V(\phi) = |\sin(2\phi)|
    \label{eq:kontrast_theorie}
\end{equation}
\cite{Hecht2002Optics}.

\subsection{Refractive Index of Glass}
\label{sec:theorie_glas}
When light beam passes through a glass plate with thickness $T$ tilted at an angle $\Theta$, the optical path length changes, which produces phase difference $\Delta \delta$
\begin{equation}
    \Delta \delta (\Theta) = \frac{2\pi}{\lambda_\text{vac}} T  \left(\frac{n - 1}{2n} \Theta^2+ \mathcal{O}(\Theta^4) \right),
    \label{eq:phasenverschiebung}
\end{equation}
where $n$ is the refractive index of the glass and $\lambda_\text{vac}$ the wavelength of the light in vacuum \cite{sample}.
If there are two plates tilted by angles $\Theta_1$ and $\Theta_2$ with $\Theta_1=-\Theta_2$, the total phase difference from equation \eqref{eq:phasenverschiebung} can be written as
\begin{align}
    \Delta \delta (\Theta) &\approx \frac{2\pi}{\lambda_\text{vac}} T \frac{n - 1}{n} \left( (\Theta_1+\Theta)^2 - (\Theta_2-\Theta)^2 \right) \nonumber \\
    & = \frac{2\pi}{\lambda_\text{vac}} T \frac{n - 1}{n} 4 \Theta_1 \cdot \Theta.
\end{align}
The number of the interference maxima or minima $M$ is proportional to the phase difference $\Delta \delta$ and can be expressed as
\begin{equation}
    M = \frac{\Delta \delta}{2\pi} = \frac{T}{\lambda_\text{vac}} \frac{n - 1}{n} 4 \Theta_1 \cdot \Theta.
    \label{eq:glas_index}
\end{equation}
So the refractive index of the glass can be obtained from equation \eqref{eq:glas_index} as
\begin{equation}
    n = \frac{1}{1 - \frac{M \lambda_\text{vac}}{4 T \Theta_1 \Theta} }.
    \label{eq:glas_index_n}
\end{equation}
It's a function of the number of interference fringes $M$, and the angles $\Theta_1$ and $\Theta$.

\subsection{Refractive Index of Air}
As seen in the previous section, when light wave passes through a medium with refractive index $n$, a phase shift arises due to the change in the optical path length.
In order to equalize the phase shift caused by a medium with refractive index $n$, the optical path length in vacuum must be adjusted to $\Delta L = \Delta n \cdot L$, where $L$ is the optical path length in the medium and $\Delta n = n - 1$.
The number of interference fringes $M$ is proportional to the phase shift $\Delta \delta$ and can be expressed as
\begin{equation}
    M = \frac{\Delta \delta}{2\pi} = \frac{\Delta n \cdot L}{\lambda_\text{vac}},
    \label{eq:gas_index}
\end{equation}
with a phase shift $\Delta \delta = \frac{2\pi}{\lambda_\text{vac}} \Delta n \cdot L$ \cite{sample}.
Thus the refractive index of gas $n$ can be obtained from equation \eqref{eq:gas_index} as
\begin{equation}
    n = 1 + \frac{M \lambda_\text{vac}}{2 \pi L}.
\end{equation}
The refractive index of a gas depends on the temperature $T$ and pressure $p$.
The Lorentz-Lorenz equation describes the relationship between the refractive index $n$ and the mean polarizability $\alpha$ of the gas:
\begin{equation}
    \frac{n^2 - 1}{n^2 + 2} = \frac{4\pi}{3} N \alpha,
    \label{eq:lorentz_lorenz}
\end{equation}
where $N$ is the number of molecules per unit volume.
With the molar refractivity $A$ and the ideal gas law $pV = nRT$, the equation \eqref{eq:lorentz_lorenz} becomes
\begin{equation}
    \frac{n^2 - 1}{n^2 + 2} = \frac{A p}{R T}.
\end{equation}
For small refractive indices ($\Delta n$ << 1), the refractive index can be approximated as
\begin{equation}
    n \approx \sqrt{1 + \frac{3 A p}{R T}}.
    \label{eq:refractive_index_gas}
\end{equation}



