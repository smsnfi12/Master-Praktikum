\section{Discussion}
\label{sec:discussion}

Comparing the results of the contrast measurement with the theoretical relations, the shape and the general tendency of the measured function
matches the theory in principle. The main differences are present when comparing the measured minima and maxima to the theoretical ones. 
\begin{table}[h!]
    \centering
    \caption{Comparison of measured and theoretical visibility values $V$.}
    \label{tab:visibility_comparison}
    \begin{tabular}{c|cc}
    \toprule
    \textbf{Measurement} & \textbf{Measured $V$} & \textbf{Theoretical $V_\text{theo}$} \\
    \midrule
    $V_\text{max,1}(45°)$ & 0.848 & 1 \\
    $V_\text{min}(90°)$   & 0.171 & 0 \\
    $V_\text{max,2}(135°)$ & 0.863 & 1 \\
    \bottomrule
    \end{tabular}
\end{table}

One reason for these deviations could be imperfect alignment of the two beams. If the two beams differ marginally from their optimal parallel alignment, 
spatial coherence is compromized, leading to different phase shifts at different points in space. Therefore, the interference would at no point be either 
absolutely constructive, nor absolutely destructive, as observed in the measured intensities. Furthermore, background radiation could lead to a residual 
intensity, measured by the diode at all times. This could explain the observation, that the minimum intensity measured, deviates further from 0, than both 
maximum intensities deviate from 1, indicating a possible positive shift of intensity by background radiation.
\newline\newline
The reflective index of glass was measured to be $n_\text{glass} = 1.529\pm0.034$, which is in alignment with the literature, specificly 
\parencite[Ch.~4, Sec.~4.3]{Hecht2002Optics}, which mentiones the reflective index of glass to be around $1.5$, depending on the type of glass. 
Regarding the measurement of the refractive index of air, the measurements show good alignment with the linear regression used in figure~\ref{fig:refrac_fit}.
In reference~\cite{MadsenEtAl2011}, the refractive index of air is mentioned to be $n\approx1.000277$ at $T = \SI{288.15}{\kelvin}$ and 
$P = \SI{101325}{\pascal}$. This aligns closely with our measurement yielding $n = n_\text{air}= 1.000264\pm0.000001$ at $T = \SI{294.85}{\kelvin}$ and 
$P = \SI{98100}{\pascal}$. 



