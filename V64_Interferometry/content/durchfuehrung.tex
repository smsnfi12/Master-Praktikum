\section{Procedure}
\label{sec:Procedure}
\subsection{Experimental Setup}
The experimental setup is shown in figure \ref{fig:aufbau}.
The light source is a He-Ne-Laser with a wavelength of $\lambda = \SI{632.99}{\nano\meter}$.
The laser emits linearly polarized light, with the polarization direction tilted by $45^\circ$ to the optical table.
The light beam is reflected by mirrors M1 and M2, which are there to align the beam path properly.
After that the beam passes through a polarizing beam-splitter cube (PBSC), which splits the beam into two orthogonally polarized beams.
The rest of the setup are three mirrors (M3, M4, M5) that guide the two beams through the interferometer, so that they overlap again at the PBSC.
In order for the two beams to interfere, they should have the same polarization state.
Therefore, a polarization filter or a second PBSC is placed in the output beam path.\\
Is the polarization filter used, so there is only one output beam. The intensity of the beam is measured with a photodiode.
However if a second PBSC, tilted by 45°, is placed in the output beam path, the PBSC splits both light beams into two orthogonally polarized beams again.
Two diodes are placed in the two output beam paths to measure the intensities of the beams.
The advantage of using two diodes is that the difference of the intensities can be measured directly, which eliminates disturbances that affect both beams equally.
This method is called differential voltage method.
\begin{figure}
    \centering
    \includegraphics[width=0.5\textwidth]{data/aufbau.png}
    \caption{Schematic setup of the Sagnac interferometer \cite{sample}.}
    \label{fig:aufbau}
\end{figure}



\subsection{Adjustment of the Interferometer}
The adjustment of the interferometer is the most crucial part of the experiment.
Without a proper alignment of the mirrors and beam-splitter cubes, no interference pattern can be observed.
The laser should hit the mirrors M1 and M2 as centrally as possible.
The mirrors M1 and M2 are adjusted so that the beam transmitted through the PBSC hits mirror M$a$ centrally.
Next, the same is done for mirror M$c$ with the deflected beam from the PBSC.
After that the mirrors are adjusted so that both beams hit the mirror M$b$ centrally.
This procedure is repeated until both beams overlap at the PBSC again and there is a single dot visible as an output.\\
Now the polarization filter is placed on the optical table to check the contrast of the interference fringes.
The interference pattern indicates that the interferometer is not perfectly aligned yet, so further adjustments are needed.
Once the interference pattern is stable and the contrast is maximized, the setup is ready for the measurements.\\
The adjustment plates on the mirrors and the PBSC are used for fine alignment, and thin plates to adjust the height of the optical components.
After the adjustment, the glass plates on the rotation holder and the gas cell can be placed in the beam path for the refractive index measurements.

\subsection{Measurements}
First, the contrast of the interferometer is measured as a function of the polarization direction of the laser beam.
There is a polarization filter placed in the beam path in front of the first PBSC.
The polarization angle $\phi$ is varied in steps of 10° from 0° to 180°.
For each angle, the maximum and minimum intensities are measured using a single photodiode.
This procedure is repeated three times to obtain more accurate results.\\
The second part is the measurement of the refractive index of glass.
For this, two tilted glass plates with a thickness of $T = \SI{1}{\milli\meter}$ are placed in one of the beam paths.
The plates are rotated from 30° to 40°, and the number of interference fringes is counted.
This procedure is repeated five times.\\
The last part is the measurement of the refraction index of air.
A gas cell with a length of $L = \SI{100\pm0.1}{\milli\meter}$ is placed on the optical table.
The cell is connected to a vacuum pump, so that the air pressure inside the cell can be varied from the lowest pressure possible up to atmospheric pressure.
The number of interference maxima or minima $M$ is counted while the pressure is increased. This procedure is repeated four times to obtain more accurate results.

